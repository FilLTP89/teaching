
% Welcome! This is the unofficial Umeå University template.
% IMPORTANT: 
% This work, "Umeå University Unofficial Beamer Theme", is a derivative
% of "University of Udine Unofficial Beamer Theme" by Marco Basaldella, University of Udine, CC 4.0 BY. 
% (https://www.overleaf.com/latex/templates/university-of-udine-unofficial-beamer-theme/zndkgxrjsdzt) 
% "Umeå University Unofficial Beamer Theme" is licensed under CC 4.0 by Jesper Erixon.

% See README.md for more full information about this template .

% Note that [usenames,dvipsnames] is MANDATORY due to compatibility
% issues between tikz and xcolor packages.

\documentclass[usenames,dvipsnames]{beamer}
%%%%
\newcommand{\mainfolder}{/home/filippo/Data/Filippo}
\newcommand{\writeup}{\mainfolder/sokrates/MSROE/CNL/lecture-notes-2019}
\newcommand{\basics}{/home/filippo/Data/Filippo/osiris/writeup/latex-basics}
\newcommand{\epstopdfimages}{\mainfolder/aeolus/eps_to_pdf/}
\newcommand{\presentations}{\writeup}
\newcommand{\runfolder}{\presentations}
\newcommand{\runsections}{\runfolder/sections}
\newcommand{\lightimages}{\runfolder/graphics/}
\newcommand{\generalimages}{/home/filippo/Data/Filippo/ares/writeup/ALL_PICTURES/images/}
\newcommand{\oldimages}{/home/filippo/Data/Filippo/virgo/writeup/ALL_PICTURES/images/}
%=====================================================
% NEWCOMMANDS
%=====================================================
\usepackage{mathtools}
\usepackage[utf8]{inputenc}
\usepackage{verbatim}
%\usepackage[english,french]{babel}
\usepackage[english]{babel}
\usepackage{lmodern}
\usepackage{latexsym}
\usepackage[normalem]{ulem}
\renewcommand{\ULdepth}{3pt}

\usepackage{amsmath,amsthm,amsfonts,amssymb,empheq,relsize,alltt}
\usepackage{xparse}
%\usepackage{algorithm}
%\usepackage{algpseudocode}
\usepackage{algorithm2e}
\usepackage{./basics/bbdd/bbdd}
\usepackage{bbm}
\usepackage[singlelinecheck=0,font={sf,small},labelfont=bf]{caption, subfig}
\captionsetup[subfigure]{subrefformat=simple,labelformat=simple,listofformat=subsimple,justification=raggedright}
\usepackage{cancel}
\usepackage{tikz}

\usepackage[noabbrev]{cleveref}
\let\chyperref\cref % Save the orginal command under a new name
\renewcommand{\cref}[1]{\hyperlink{#1}{\chyperref{#1}}} % Redefine the \cref command and explictely add the hyperlink. 
%\usepackage[T1]{fontenc}
\usepackage{media9}
\usepackage{multimedia}
\usepackage{animate}
\usepackage{stmaryrd}
%\usepackage{minted}

\newenvironment{questions}{\begin{list}{\arabic{nquestion}.}{\usecounter{nquestion}\setlength{\itemindent}{-0.5pt}\setlength{\listparindent}{250pt}}}{\unskip\end{list}}
%
\newcommand{\myparagraph}[1]{\subsubsection*{#1}}
% BEAMER
\newcommand{\ptl}[2]{{\large \textcolor{#2}{\textbf{#1}}}}
\newcommand{\cit}[1]{{\tiny \textbf{#1}}}
% TEXTBLOCK
\newcommand{\txb}[4]{
\begin{textblock}{#1}(#2,#3)
	#4
\end{textblock}
}
\newcommand{\blk}[2]{
\begin{block}{#1}
	#2
\end{block}
}
\newcommand{\eblk}[2]{
\begin{exampleblock}{#1}
	#2
\end{exampleblock}
}
\newcommand{\ablk}[2]{
\begin{alertblock}{#1}
	#2
\end{alertblock}
}

\newcommand{\frm}[2]{
\begin{frame}{#1}
	#2
\end{frame}
}
\newcommand{\itmz}[1]{
\begin{itemize}
	#1
\end{itemize}
}
% INCLUDE GRAPHICS WIDTH
\newcommand{\igwf}[2]{
\includegraphics[keepaspectratio,width=#2]{#1}
}
% INCLUDE GRAPHICS HEIGHT
\newcommand{\ighf}[2]{
	\includegraphics[keepaspectratio,height=#2]{#1}
}
% SUBFIGURE
\newcommand{\subf}[2]{
\subfloat[#2]{#1\label{fig:#1}}
}
% SUBFIGURE WIDTH
\newcommand{\subfw}[3]{
	\subfloat[#3]{\igwf{#1}{#2}\label{fig:#1}}
}
% SUBFIGURE HEIGHT
\newcommand{\subfh}[3]{
	\subfloat[#3]{\ighf{#1}{#2}\label{fig:#1}}
}
% BEGIN FIGURE
\newcommand{\bfc}[3]{
\begin{figure}[ht!]
	\centering
	#1
	\caption{#2\label{fig:#3}}
\end{figure}
}
% BEGIN WRAPPED FIGURE
\usepackage{wrapfig}
\usepackage{capt-of}

\DeclareDocumentCommand{\bwfc}{ O{} O{} O{} O{l} O{20mm}}{
	\begin{wrapfigure}{#4}{#5}
		\begin{center}
			#1
			\captionof{figure}{#2}\label{fig:#3}
		\end{center}
\end{wrapfigure}
}

% EQUATION
\newcommand{\beq}[2]{
\begin{equation}
	#1
	\label{eq:#2}
\end{equation}
}
\DeclareDocumentCommand{\beqo}{ O{} O{} O{} }{
	\begin{equation}
	#1
	\label#3{eq:#2}
	\end{equation}
}
\DeclareDocumentCommand{\beqh}{ O{} O{} O{}  O{} }{
	\begin{equation}
	#1
	\hypertarget#3{#4}{\label#3{eq:#2}}
	\end{equation}
}


\DeclareDocumentCommand{\bsys}{ O{} O{} O{} }{
	\begin{empheq}[left=#1\empheqlbrace]{align}
	#2
	\label{eq:#3}
	\end{empheq}
}

\DeclareDocumentCommand{\bmat}{ O{} O{}  }{
	  \begin{bmatrix}
	  	\begin{array}{#2}
		#1
	\end{array}
	\end{bmatrix}
}
\DeclareDocumentCommand{\barr}{ O{} }{
	\begin{array}{c|c}
		#1
	\end{array}
}
%\usepackage{keyval}
%\makeatletter
%\define@key{janbertdims}{l}{\def\janbert@l{#1}}
%\define@key{janbertdims}{w}{\def\janbert@w{#1}}
%\define@key{janbertdims}{h}{\def\janbert@h{#1}}
%% initialize; change here to your preferred values
%\setkeys{janbertdims}{
%	l=1,        
%	w=1,    
%	h=1,
%}
%
%\newcommand\dimsEN[1][]{%
%	\begingroup
%	\setkeys{janbertdims}{#1}% the current values
%	The dimensions ($\mathrm{L} \times \mathrm{W} \times \mathrm{H}$)
%	of the room are
%	$\janbert@l \times \janbert@w \times \janbert@h$%
%	\endgroup
%}
%\makeatother
%
%	\dimsEN
%	
%	\dimsEN[h=3]
%	
%	\dimsEN[h=4,w=2]
%	
%	\dimsEN[w=3,l=5,h=2]
% GGD
\newcommand{\gggd}{$\frac{\text{G}}{\text{G}_\text{max}}-\gamma-\text{D}$ curves }
% KKNPP
\newcommand{\kka}[1][Nuclear Power Plant]{Kashiwazaki-Kariwa #1 }
% NCO
\newcommand{\NCO}{ \textit{Niigata-Ken Ch\={u}etsu-Oki} }
% TEPCO
\newcommand{\TEPCO}{ \textit{Tokyo Electric Power Company} }
% VS/VP/VS30
\newcommand{\vsth}{ V$_\text{S,30}$ }
\newcommand{\vs}{$V_{S}$ }
\newcommand{\vp}{$V_{P}$ }

\newcommand{\N}{\mathbb{N}}
\newcommand{\Z}{\mathbb{Z}}
%\newcommand{\C}{\mathbb{C}}
\DeclareDocumentCommand{\opercal}{O{}O{}O{}}{\ensuremath{\mathcal{#1}_{#2}^{#3}}}
\DeclareDocumentCommand{\operbb}{O{}O{}O{}}{\ensuremath{\mathbb{#1}_{#2}^{#3}}}
% VECTOR
\newcommand{\vct}[3]{\ensuremath{\boldsymbol{\underline{#1}}_{#2}^{#3} }}
\DeclareDocumentCommand{\ebv}{ O{} O{} O{} }{\vct{e}{#1}{#2}#3}
\DeclareDocumentCommand{\Phibv}{ O{} O{} O{} }{\vct{\Phi}{#1}{#2}#3}
\DeclareDocumentCommand{\ibv}{ O{} O{} O{} }{\vct{i}{#1}{#2}#3}

\DeclareDocumentCommand{\ione}{ O{} O{} }{\vct{i}{1}{#1}#2}
\DeclareDocumentCommand{\itwo}{ O{} O{} }{\vct{i}{2}{#1}#2}
\DeclareDocumentCommand{\ithree}{ O{} O{} }{\vct{i}{3}{#1}#2}
\DeclareDocumentCommand{\ix}{ O{} O{} }{\vct{i}{x}{#1}#2}
\DeclareDocumentCommand{\iy}{ O{} O{} }{\vct{i}{y}{#1}#2}

\DeclareDocumentCommand{\ir}{ O{} O{} O{} }{\vct{i}{r_{#1}}{#2}#3}
\DeclareDocumentCommand{\ith}{ O{} O{} O{} }{\vct{i}{\theta_{#1}}{#2}#3}
\DeclareDocumentCommand{\iz}{ O{} O{} O{} }{\vct{i}{z_{#1}}{#2}#3}
% beams
\DeclareDocumentCommand{\eXone}{ O{} }{\ebv[\chi_1]#1}
\DeclareDocumentCommand{\eXtwo}{ O{} }{\ebv[\chi_2]#1}
\DeclareDocumentCommand{\pG}{O{G}}{\pv[#1]}
\DeclareDocumentCommand{\pGp}{O{G} O{\textquotesingle}}{\pv[#1][#2]}
\DeclareDocumentCommand{\psigma}{O{\Sigma}}{\pv[#1]}
\DeclareDocumentCommand{\xG}{O{G}}{\xv[#1]}
\DeclareDocumentCommand{\xGp}{O{G} O{\textquotesingle}}{\xv[#1][#2]}
\DeclareDocumentCommand{\xsigma}{O{\Sigma}}{\xv[#1]}
\DeclareDocumentCommand{\uG}{O{G}}{\uv[#1]}
\DeclareDocumentCommand{\uGp}{O{G} O{\textquotesingle}}{\uv[#1][#2]}
\DeclareDocumentCommand{\uGe}{O{Ge} O{}}{u_{#1}#2}
\DeclareDocumentCommand{\uGep}{O{Ge} O{\textquotesingle}}{u_{#1}^{#2}}
\DeclareDocumentCommand{\uGepp}{O{Ge} O{\textquotesingle\textquotesingle}}{u_{#1}^{#2}}

\DeclareDocumentCommand{\usigma}{O{\Sigma}}{\uv[#1]}
\DeclareDocumentCommand{\uGsigma}{O{G\Sigma}}{\uv[#1]}
\DeclareDocumentCommand{\uGsigmap}{O{G\Sigma} O{\textquotesingle}}{\uv[#1][#2]}
\DeclareDocumentCommand{\uGsigmapp}{O{G\Sigma} O{\textquotesingle\textquotesingle}}{\uv[#1][#2]}

\DeclareDocumentCommand{\uGperp}{O{G\vdash}}{\uv[#1]}
\DeclareDocumentCommand{\uGperpp}{O{G\vdash} O{\textquotesingle}}{\uv[#1][#2]}
\DeclareDocumentCommand{\uGperppp}{O{G\vdash} O{\textquotesingle\textquotesingle}}{\uv[#1][#2]}

\DeclareDocumentCommand{\thetam}{ O{} O{} O{} }{\tens{\theta}{#1}{\wedge}#3}
\DeclareDocumentCommand{\thetav}{ O{} O{} O{} }{\vct{\theta}{#1}{}#3}
\DeclareDocumentCommand{\thetavp}{ O{} O{} }{\vct{\theta}{#1}{\textquotesingle}#2}
\DeclareDocumentCommand{\thsigmap}{ O{} }{\vct{\theta}{\Sigma}{\textquotesingle}#1}
\DeclareDocumentCommand{\thsigmapp}{ O{} }{\vct{\theta}{\Sigma}{\textquotesingle\textquotesingle}#1}
\DeclareDocumentCommand{\thsigma}{ O{\Sigma} O{} O{} }{\vct{\theta}{#1}{#2}#3}
\DeclareDocumentCommand{\thetae}{ O{e} O{} }{\theta_{#1}#2}
\DeclareDocumentCommand{\thetaep}{ O{e} O{\textquotesingle} }{\theta_{#1}^{#2}}
\DeclareDocumentCommand{\thetavperp}{ O{\vdash} O{} }{\vct{\theta}{\vdash}{#2}}
\DeclareDocumentCommand{\dduGv}{ O{} O{} O{} }{\vct{\ddot{u}}{G}{#2}#3}
\DeclareDocumentCommand{\dduGsigma}{ O{} O{} O{} }{\vct{\ddot{u}}{G\Sigma}{#2}#3}
\DeclareDocumentCommand{\ddthetav}{ O{} O{} O{} }{\vct{\ddot{\theta}}{#1}{#2}#3}
\DeclareDocumentCommand{\ddthetae}{}{\ensuremath{\ddot{\theta}_e}}
\DeclareDocumentCommand{\pthp}{ O{} }{\ensuremath{\pdp{\theta_{#1}}} }
\DeclareDocumentCommand{\stressref}{O{ref}}{\ensuremath{\stress \phantom{}_{\text{#1}}}}

\DeclareDocumentCommand{\hv}{ O{} O{} O{} }{\vct{h}{#1}{#2}#3}
\DeclareDocumentCommand{\ov}{ O{} O{} O{} }{\vct{o}{#1}{#2}#3}
\DeclareDocumentCommand{\yv}{ O{} O{} O{} }{\vct{y}{#1}{#2}#3}
\DeclareDocumentCommand{\pv}{ O{} O{} O{} }{\vct{p}{#1}{#2}#3}
\DeclareDocumentCommand{\Xv}{ O{} O{} O{} }{\vct{X}{#1}{#2}#3}
\DeclareDocumentCommand{\xv}{ O{} O{} O{} }{\vct{x}{#1}{#2}#3}
\DeclareDocumentCommand{\xgv}{ O{} O{} O{} }{\vct{x}{g}{#1}#2}
\DeclareDocumentCommand{\ddxgv}{ O{} O{} O{} }{\vct{\ddot{x}}{g}{#1}#2}
\DeclareDocumentCommand{\tv}{ O{} O{} O{} }{\vct{t}{#1}{#2}#3}
\DeclareDocumentCommand{\fv}{ O{} O{} O{} }{\vct{f}{#1}{#2}#3}
\DeclareDocumentCommand{\Ft}{ O{} O{} O{} }{\tens{F}{#1}{#2}#3}
\DeclareDocumentCommand{\cv}{ O{} O{} O{} }{\vct{c}{#1}{#2}#3}
\DeclareDocumentCommand{\Ct}{ O{} O{} O{} }{\tens{C}{#1}{#2}#3}
\DeclareDocumentCommand{\dv}{ O{} O{} O{} }{\vct{d}{#1}{#2}#3}
\DeclareDocumentCommand{\Dt}{ O{} O{} O{} }{\tens{D}{#1}{#2}#3}
\DeclareDocumentCommand{\vV}{ O{} O{} O{} }{\vct{v}{#1}{#2}#3}
\DeclareDocumentCommand{\bv}{ O{} O{} O{} }{\vct{b}{#1}{#2}#3}
\DeclareDocumentCommand{\Bt}{ O{} O{} O{} }{\tens{B}{#1}{#2}#3}
\DeclareDocumentCommand{\Vv}{ O{} O{} O{} }{\vct{V}{#1}{#2}#3}
\DeclareDocumentCommand{\qv}{ O{} O{} O{} }{\vct{q}{#1}{#2}#3}
\DeclareDocumentCommand{\Qt}{ O{} O{} O{} }{\tens{Q}{#1}{#2}#3}
\DeclareDocumentCommand{\qvF}{ O{} O{} O{} }{\vct{\hat{q}}{#1}{#2}#3}
\DeclareDocumentCommand{\ddqv}{ O{} O{} O{} }{\vct{\ddot{q}}{#1}{#2}#3}
\DeclareDocumentCommand{\av}{ O{} O{} O{} }{\vct{a}{#1}{#2}#3}
\DeclareDocumentCommand{\avf}{ O{} O{} O{} }{\vct{\tilde{a}}{#1}{#2}#3}
\DeclareDocumentCommand{\at}{ O{} O{} O{} }{\tens{a}{#1}{#2}#3}
\DeclareDocumentCommand{\At}{ O{} O{} O{} }{\tens{A}{#1}{#2}#3}
\DeclareDocumentCommand{\Av}{ O{} O{} O{} }{\vct{A}{#1}{#2}#3}
\DeclareDocumentCommand{\wv}{ O{} O{} O{} }{\vct{w}{#1}{#2}#3}
\DeclareDocumentCommand{\zv}{ O{} O{} O{} }{\vct{z}{#1}{#2}#3}
\DeclareDocumentCommand{\fm}{ O{} O{} O{} }{\vct{f}{#1}{#2}#3}
\DeclareDocumentCommand{\nv}{ O{} O{} O{} }{\vct{n}{#1}{#2}#3}
\DeclareDocumentCommand{\Nv}{ O{} O{} O{} }{\vct{N}{#1}{#2}#3}
\DeclareDocumentCommand{\Bv}{ O{} O{} O{} }{\vct{B}{#1}{#2}#3}
\DeclareDocumentCommand{\mv}{ O{} O{} O{} }{\vct{m}{#1}{#2}#3}
\DeclareDocumentCommand{\qv}{ O{} O{} O{} }{\vct{q}{#1}{#2}#3}
\DeclareDocumentCommand{\gv}{ O{} O{} O{} }{\vct{g}{#1}{#2}#3}
\DeclareDocumentCommand{\tauv}{ O{} O{} O{} }{\vct{\tau}{#1}{#2}#3}
\DeclareDocumentCommand{\tauvsigma}{ O{} O{} O{} }{\vct{\tau}{\Sigma}{#2}#3}
\DeclareDocumentCommand{\Nv}{ O{} O{} O{} }{\vct{N}{#1}{#2}#3}
\DeclareDocumentCommand{\Nm}{ O{} O{} O{} }{\tens{N}{#1}{#2}#3}
\DeclareDocumentCommand{\Rv}{ O{} O{} O{} }{\vct{R}{#1}{#2}#3}
\DeclareDocumentCommand{\Rm}{ O{} O{} O{} }{\tens{R}{#1}{#2}#3}
\DeclareDocumentCommand{\Mv}{ O{} O{} O{} }{\vct{M}{#1}{#2}#3}
\DeclareDocumentCommand{\Fv}{ O{} O{} O{} }{\vct{F}{#1}{#2}#3}
\DeclareDocumentCommand{\nvsigma}{ O{} }{\vct{n}{\Sigma}{}}
\DeclareDocumentCommand{\xiv}{ O{} O{} O{} }{\vct{xi}{#1}{#2}#3}
\DeclareDocumentCommand{\uv}{ O{} O{} O{} }{\vct{u}{#1}{#2}#3}
%\DeclareDocumentCommand{\vv}{ O{} O{} O{} }{\vct{v}{#1}{#2}#3}
\DeclareDocumentCommand{\Uv}{ O{} O{} O{} }{\vct{U}{#1}{#2}#3}
\DeclareDocumentCommand{\unx}{ O{} }{\vct{\hat{u}}{n}{#1}\pdp{\xv}}
\DeclareDocumentCommand{\duv}{ O{} O{} O{} }{\vct{\dot{u}}{#1}{#2}#3}
\DeclareDocumentCommand{\dduv}{ O{} O{} O{} }{\vct{\ddot{u}}{#1}{#2}#3}
\DeclareDocumentCommand{\psiv}{ O{} O{} O{} }{\vct{\psi}{#1}{#2}#3}
\DeclareDocumentCommand{\phiv}{ O{} O{} O{} }{\vct{\phi}{#1}{#2}#3}
\DeclareDocumentCommand{\Phiv}{ O{} O{} O{} }{\vct{\Phi}{#1}{#2}#3}
\DeclareDocumentCommand{\phim}{ O{} O{} O{} }{\tens{\Phi}{#1}{#2}#3}
\DeclareDocumentCommand{\vphiv}{ O{} O{} O{} }{\vct{\varphi}{#1}{#2}#3}

\newcommand{\zerov}{\vct{0}{}{}}
\newcommand{\zerot}{\tens{0}{}{}}
% PARENTHESIS
\newcommand{\pdp}[1]{\ensuremath{\left(#1\right)}}
\newcommand{\psp}[1]{\ensuremath{\left[#1\right]}}
\newcommand{\pcp}[1]{\ensuremath{\langle#1\rangle}}
\newcommand{\pbp}[1]{\ensuremath{\left\lbrace #1\right\rbrace }}
\newcommand{\uasr}[3]{\ensuremath{\left(#1_{#2}^{#3}\right)}}
\newcommand{\uavr}[3]{\ensuremath{\left(\vct{#1}{#2}{#3}\right)}}
\newcommand{\uavb}[3]{\ensuremath{\llbracket\vct{#1}{#2}{#3}\rrbracket}}
\newcommand{\disc}[1]{\ensuremath{\llbracket #1\rrbracket}}
\newcommand{\dassr}[6]{\ensuremath{\left(#1_{#2}{#3},#4_{#5}{#6}\right)}}
\newcommand{\davvr}[6]{\ensuremath{\left(\vct{#1}{#2}{#3},\vct{#4}{#5}{#6}\right)}}
\newcommand{\davsr}[6]{\ensuremath{\left(\vct{#1}{#2}{#3};#4_{#5}^{#6}\right)}}
\newcommand{\davtvtg}[9]{%
	\def\tempa{#1}%
	\def\tempb{#2}%
	\def\tempc{#3}%
	\def\tempd{#4}%
	\def\tempe{#5}%
	\def\tempf{#6}%
	\def\tempg{#7}%
	\def\temph{#8}%
	\def\tempi{#9}%
	\davtvtgcontinued
}
\newcommand{\davtvtgcontinued}[3]{%
	\ensuremath{\left(\vct{\tempa}{\tempb}{\tempc},\tempd_{\tempe}^{\tempf}\vert\vct{\tempg}{\temph}{\tempi},#1_{#2}^{#3}\right)}
}
\DeclareDocumentCommand{\xiyi}{O{x} O{y} O{i} O{i}}{\davvr{#1}{#3}{}{#2}{#4}{}}
\DeclareDocumentCommand{\xifxi}{O{x} O{f} O{i} O{i}}{\pdp{\vct{#1}{#3}{},\vct{#2}{}{}\pdp{\vct{#1}{#4}{}} }}


\newcommand{\pxyts}{\ensuremath{\left(\xv,t;\xiv,\tau\right)}}
%
\newcommand{\pxtp}{\ensuremath{\davsr{x}{}{}{t}{}{}}}
\newcommand{\pXtp}{\ensuremath{\davsr{X}{}{}{t}{}{}}}
\newcommand{\pptp}{\ensuremath{\davsr{p}{}{}{t}{}{}}}
\newcommand{\pdxtp}[1]{\ensuremath{\left(#1;t\right)}}
\newcommand{\pxdtp}[1]{\ensuremath{\left(\vct{x}{}{};#1\right)}}
\newcommand{\pxop}{\ensuremath{\davsr{x}{}{}{\omega}{}{}}}
\newcommand{\pcop}{\ensuremath{\davsr{\xi}{}{}{\omega}{}{}}}
\newcommand{\pctp}{\ensuremath{\davsr{x}{}{}{t}{}{}}}
\newcommand{\pxp}{\ensuremath{\pdp{\xv}}}
\newcommand{\pyp}{\ensuremath{\pdp{\yv}}}
\newcommand{\pxyp}{\ensuremath{\davvr{x}{}{}{y}{}{}}}

\newcommand{\pscp}{\ensuremath{\pdp{\stress;\chih}}}
% TENSORS otimes products
\DeclareDocumentCommand{\otp}{ O{ } O{ } }{\ensuremath{#1\otimes #2}}
\DeclareDocumentCommand{\sotp}{ O{ } O{ } }{\ensuremath{#1\otimes_S#2}}
\DeclareDocumentCommand{\aotp}{ O{ } O{ } }{\ensuremath{#1\otimes_A#2}}
%\NewDocumentCommand{\qfrac}{smm}{%
%	\dfrac{\IfBooleanT{#1}{\vphantom{\big|}}#2}{\mathstrut #3}%
%}

\newcommand{\tens}[3]{\ensuremath{\uuline{\boldsymbol{#1}}\phantom{}_{#2}^{#3}}}
\newcommand{\tenst}[3]{\ensuremath{\uuuline{\boldsymbol{#1}}_{#2}^{#3}}}

\newcommand{\tensdot}[3]{\ensuremath{\uuline{\boldsymbol{\dot{#1}}}_{#2}^{#3}}}
\newcommand{\tpa}{\otimes_a}
\newcommand{\tps}{\otimes_s}
\DeclareDocumentCommand{\Jin}{ O{} O{} }{\ensuremath{\tens{\mathbb{J}_{#1}}{}{#2}}}

\DeclareDocumentCommand{\Jinm}{ O{\rho} }{
	\ensuremath{\tens{\mathbb{J}_{}}{}{}\phantom{}_{#1}}
}
\DeclareDocumentCommand{\stress}{ O{} O{} }{\ensuremath{\tens{\dsigma}{#1}{#2}}}
\DeclareDocumentCommand{\stressdot}{ O{} O{} }{\ensuremath{\tensdot{\dsigma}{#1}{#2}}}
\DeclareDocumentCommand{\dstress}{ O{} O{} }{\ensuremath{\tens{\delta\dsigma}{#1}{#2}}}
\newcommand{\stressnn}{\dsigma_{nn}}
\newcommand{\stressm}{\dsigma_{m}}
\DeclareDocumentCommand{\strain}{ O{} O{}}{\ensuremath{\tens{\depsil}{#1}{#2}}}
\DeclareDocumentCommand{\straindot}{ O{} O{}}{\ensuremath{\tens{\dot{\depsil}}{#1}{#2}}}
\DeclareDocumentCommand{\dstrain}{ O{} O{}}{\ensuremath{\tens{\delta\depsil}{#1}{#2}}}
\DeclareDocumentCommand{\strainel}{O{}}{\ensuremath{\strain[#1][el]}}

\DeclareDocumentCommand{\deviator}{ O{} O{} }{\tens{s}{#1}{\dsigma#2}}

\newcommand{\deps}{\tensdot{\depsil}{x}{}}

\newcommand{\strainpl}{\ensuremath{\strain[][pl]}}
\newcommand{\straineldot}{\ensuremath{\straindot[][el]}}
\newcommand{\strainpldot}{\ensuremath{\straindot[][pl]}}
\newcommand{\dstrainel}{\ensuremath{\dstrain[][el]}}
\newcommand{\dstrainpl}{\ensuremath{\dstrain[][pl]}}

\DeclareDocumentCommand{\edv}{O{}}{\ensuremath{\tens{e}{#1}{}}}
\DeclareDocumentCommand{\epl}{O{}}{\ensuremath{\tens{e}{#1}{pl}}}
\DeclareDocumentCommand{\eel}{O{}}{\ensuremath{\tens{e}{#1}{el}}}
\DeclareDocumentCommand{\evol}{O{}}{\ensuremath{\depsil_{vol}^{#1}}}
\DeclareDocumentCommand{\devol}{O{}}{\ensuremath{\delta\depsil_{vol}^{#1}}}
\DeclareDocumentCommand{\evoldot}{O{}}{\ensuremath{\dot{\depsil}_{vol}^{#1}}}

\newcommand{\evolel}{\ensuremath{\evol[el]}}
\newcommand{\evolpl}{\ensuremath{\evol[pl]}}
\DeclareDocumentCommand{\devolel}{O{}}{\ensuremath{\delta\depsil_{vol}^{el}}}
\DeclareDocumentCommand{\evoldotel}{O{}}{\ensuremath{\dot{\depsil}_{vol}^{el}}}

\DeclareDocumentCommand{\evolpl}{O{}}{\ensuremath{\depsil_{vol}^{pl}}}
\DeclareDocumentCommand{\devolpl}{O{}}{\ensuremath{\delta\depsil_{vol}^{pl}}}
\DeclareDocumentCommand{\evoldotpl}{O{}}{\ensuremath{\dot{\depsil}_{vol}^{pl}}}
\DeclareDocumentCommand{\ebar}{O{} O{}}{\ensuremath{\bar{e}{#1}^{#2}}}
\DeclareDocumentCommand{\edev}{O{}}{\ensuremath{\depsil_d^{#1}}}
\DeclareDocumentCommand{\dedev}{O{}}{\ensuremath{\delta\depsil_d^{#1}}}
\newcommand{\dedevel}{\ensuremath{\delta\depsil_d^{el}}}
\newcommand{\dedevpl}{\ensuremath{\delta\depsil_d^{pl}}}
\DeclareDocumentCommand{\edevdot}{O{}}{\ensuremath{\dot{\depsil}_d^{#1}}}
\DeclareDocumentCommand{\edvdot}{O{}}{\ensuremath{\tensdot{e}{#1}{el}}}
\DeclareDocumentCommand{\edvpldot}{O{}}{\ensuremath{\tensdot{e}{#1}{pl}}}
\DeclareDocumentCommand{\edveldot}{O{}}{\ensuremath{\tensdot{e}{#1}{el}}}

\DeclareDocumentCommand{\dedv}{O{}}{\ensuremath{\tens{\delta e}{#1}{}}}
\DeclareDocumentCommand{\dedvpl}{O{}}{\ensuremath{\tens{\delta e}{#1}{pl}}}
\DeclareDocumentCommand{\dedvel}{O{}}{\ensuremath{\tens{\delta e}{#1}{el}}}

\DeclareDocumentCommand{\fvm}{O{}}{\ensuremath{\sqrt{3J_{2}\pdp{\deviator-\tens{X}{}{}#1}}}}

\newcommand{\dq}{\ensuremath{\delta q}}
\newcommand{\dpeff}{\ensuremath{\delta p'}}
\newcommand{\Deppp}{\ensuremath{\mathsf{D}_{p'p'}^{ep}}}
\newcommand{\Deppq}{\ensuremath{\mathsf{D}_{p'q}^{ep}}}
\newcommand{\Depqp}{\ensuremath{\mathsf{D}_{qp'}^{ep}}}
\newcommand{\Depqq}{\ensuremath{\mathsf{D}_{qq}^{ep}}}
\newcommand{\Ceppp}{\ensuremath{\mathsf{C}_{p'p'}^{ep}}}
\newcommand{\Ceppq}{\ensuremath{\mathsf{C}_{p'q}^{ep}}}
\newcommand{\Cepqp}{\ensuremath{\mathsf{C}_{qp'}^{ep}}}
\newcommand{\Cepqq}{\ensuremath{\mathsf{C}_{qq}^{ep}}}
\newcommand{\Deptriax}{\ensuremath{\bmat[\Deppp & \Deppq\\ \hline \Depqp & \Depqq][c|c]}}
\newcommand{\Ceptriax}{\ensuremath{\bmat[\Ceppp & \Ceppq\\ \hline \Cepqp & \Cepqq][c|c]}}

\newcommand{\Delpp}{\ensuremath{\mathsf{D}_{p'p'}^{el}}}
\newcommand{\Delpq}{\ensuremath{\mathsf{D}_{p'q}^{el}}}
\newcommand{\Delqp}{\ensuremath{\mathsf{D}_{qp'}^{el}}}
\newcommand{\Delqq}{\ensuremath{\mathsf{D}_{qq}^{el}}}
\newcommand{\Celpp}{\ensuremath{\mathsf{C}_{p'p'}^{el}}}
\newcommand{\Celpq}{\ensuremath{\mathsf{C}_{p'q}^{el}}}
\newcommand{\Celqp}{\ensuremath{\mathsf{C}_{qp'}^{el}}}
\newcommand{\Celqq}{\ensuremath{\mathsf{C}_{qq}^{el}}}
\newcommand{\Deltriax}{\ensuremath{\bmat[\Delpp & 0\\ \hline 0 & \Delqq][c|c]}}
\newcommand{\Celtriax}{\ensuremath{\bmat[\Celpp & 0\\ \hline 0 & \Celqq][c|c]}}

\newcommand{\triaxdstress}{\ensuremath{\bmat[ \dpeff\\	\dq][c]}}
\newcommand{\triaxdstrain}{\ensuremath{\bmat[\devol\\ \dedev][c]}}
\newcommand{\uw}{\ensuremath{u_w}}
\newcommand{\duw}{\ensuremath{\delta u_w}}
\newcommand{\uwdot}{\ensuremath{\dot{u}_w}}

\DeclareDocumentCommand{\backstress}{O {} O {} O{}}{\ensuremath{\tens{X}{}{}}}
\DeclareDocumentCommand{\kbackstress}{O {} O {} O{}}{\ensuremath{\tens{\dalpha}{}{}}}
\DeclareDocumentCommand{\dbackstress}{O {} O {} O{}}{\ensuremath{\tens{\delta X}{}{}}}
\DeclareDocumentCommand{\dkbackstress}{O {} O {} O{}}{\ensuremath{\tens{\delta\dalpha}{}{}}}

\newcommand{\norm}[1]{\ensuremath{\Vert #1 \Vert}}
\newcommand{\normv}[3]{\ensuremath{\Vert \vct{#1}{#2}{#3} \Vert}}
\newcommand{\ID}{\ensuremath{\tens{I}{}{}}}
\newcommand{\IDD}{\ensuremath{\tensf{I}{}{}}}
\newcommand{\RotR}{\ensuremath{\tens{\mathbb{R}}{}{}}}
\DeclareDocumentCommand{\RotQ}{ O{} O{} O{} O{}}{ \ensuremath{\tens{\mathbb{Q}}{}{}\davsr{#1}{#2}{}{#3}{}{}}
}
\DeclareDocumentCommand{\Pt}{ O{} O{} }{\ensuremath{\tens{P}{#1}{#2}}}
% TENSORS HIGHER ORDER

\newcommand{\tensf}[3]{\ensuremath{\mathlarger{\mathlarger{\boldsymbol{\mathsf{#1}_{#2}^{#3}}}}}}
\newcommand{\Del}{\ensuremath{\tensf{D}{}{el}}}
\newcommand{\Cel}{\ensuremath{\tensf{C}{}{el}}}
\newcommand{\Dve}{\ensuremath{\mathlarger{\mathcal{D}}^{ve}}}
\newcommand{\Dep}{\ensuremath{\tensf{D}{}{ep}}}
\newcommand{\Cep}{\ensuremath{\tensf{C}{}{ep}}}
\newcommand{\Deps}{\tens{\Delta \varepsilon}{n}{}}
\newcommand{\Depspl}{\tens{\Delta \varepsilon}{}{pl}}
\newcommand{\Depsel}{\tens{\Delta \varepsilon}{}{el}}
%
\newcommand{\Dvel}{\ensuremath{\mathlarger{\mathbf{D}}^{el}}}
\newcommand{\Dvep}{\ensuremath{\mathlarger{\mathbf{D}}^{ep}}}
\newcommand{\Dveps}{\vct{\Delta \varepsilon}{n}{}}
\newcommand{\Dvepspl}{\vct{\Delta \varepsilon}{}{pl}}
\newcommand{\Dvepsel}{\vct{\Delta \varepsilon}{}{el}}
%
\newcommand{\Dsigt}[1][trial]{\tens{\Delta \sigma}{n}{#1}}
\newcommand{\Trt}[1]{\ensuremath{Tr\left(#1\right)}}
\newcommand{\dco}[1]{\ensuremath{#1:#1}}
\newcommand{\dcom}[2]{\ensuremath{#1:#2}}

\newcommand{\hidd}[3]{\ensuremath{\underset{\sim}{\smash{\boldsymbol{#1}}}_{#2}^{#3}}}
\newcommand{\hiddot}[3]{\ensuremath{\underset{\sim}{\smash{\boldsymbol{\dot{#1}}}}_{#2}^{#3}}}

\DeclareDocumentCommand{\etah}{ O{} O{} }{\ensuremath{\hidd{\eta}{#1}{#2}}}
\DeclareDocumentCommand{\detah}{ O{} O{} }{\ensuremath{\hidd{\delta\eta}{#1}{#2}}}

\DeclareDocumentCommand{\chih}{ O{} O{} }{\ensuremath{\hidd{\dchi}{#1}{#2}}}
\DeclareDocumentCommand{\chihdot}{ O{} O{} }{\ensuremath{\hidddot{\chi}{#1}{#2}}}
\DeclareDocumentCommand{\dchih}{ O{} O{} }{\ensuremath{\hidd{\delta\dchi}{#1}{#2}}}

\DeclareDocumentCommand{\nablav}{O{x} O{}}{\ensuremath{\vct{\nabla}{#1}{#2}}}
\newcommand{\gradv}[2]{\ensuremath{\vct{\nabla}{#2}{}#1}}
\newcommand{\gradt}[2]{\ensuremath{\tens{\mathbb{D}_{#2}#1}{}{}}}
\newcommand{\gradtT}[2]{\ensuremath{\tens{\mathbb{D}_{#2}^T#1}{}{}}}
\newcommand{\gradtS}[2]{\ensuremath{\tens{\mathbb{D}_{#2}^S#1}{}{}}}
\newcommand{\gradh}[2]{\ensuremath{\underset{\sim}{\mathbb{D}_{#2}#1}}}
\newcommand{\gradtt}[5]{\ensuremath{\vct{#1}{#2}{#3}\otimes\vct{\nabla}{#4}{#5}}}
\newcommand{\gradtts}[5]{\ensuremath{\vct{\nabla}{#4}{#5}\otimes\vct{#1}{#2}{#3}}}
\newcommand{\gradttsym}[5]{\ensuremath{\vct{#1}{#2}{#3}\otimes_s\vct{\nabla}{#4}{#5}}}
% scalar product
\newcommand{\pscl}[2]{\ensuremath{\left\langle#1,#2\right\rangle}}
\DeclareDocumentCommand{\dive}{ O{x} }{\ensuremath{\vct{\nabla}{#1}{}.}}

\newcommand{\roto}[2]{\ensuremath{\vct{\nabla}{#2}{}\wedge #1}}
\newcommand{\lapl}[2]{\ensuremath{\Delta_{#2}#1}}
\newcommand{\laplv}[4]{\ensuremath{\underline{\boldsymbol{\Delta}_{#4}\boldsymbol{#1_{#2}^{#3}}}}}
%
% DERIVATIVES
\DeclareDocumentCommand{\matd}{O{} O{t}}{\ensuremath{\frac{D#1}{D#2}}}

\newcommand{\dts}{\ensuremath{\partial_{tt}}}
\newcommand{\dtp}{\ensuremath{\partial_{t}}}
\DeclareDocumentCommand{\pard}{O{} O{t}}{\ensuremath{\frac{\partial #1}{\partial #2}}}
\newcommand{\ftp}[1]{\ensuremath{\widehat{#1}}}
\newcommand{\fts}[1]{\ensuremath{\widehat{\widehat{#1}}}}
%
\newcommand{\ff}[2]{\ensuremath{#1\left(#2\right)}}
\newcommand{\dd}[2]{\ensuremath{\frac{\partial #1}{\partial #2}}}
\newcommand{\ev}[2]{\ensuremath{#1\Bigr|_{#2}}}
\DeclareDocumentCommand{\partdt}{ O{} O{x} }{\ensuremath{\frac{\partial #1}{\partial t}} }

\newcommand{\scp}[2]{\ensuremath{\left(#1,#2\right)}}
\newcommand{\abs}[1]{\ensuremath{\vert #1 \vert}}
\newcommand{\pdf}[2]{\ensuremath{\frac{\partial#1}{\partial#2}}}
\newcommand{\mdf}[2]{\ensuremath{\frac{d#1}{d#2}}}
\newcommand{\sdf}[2]{\ensuremath{\frac{\partial^2#1}{\partial#2^2}}}
\newcommand{\nmdf}[3]{\ensuremath{\frac{d^{#3}#1}{d#2}}}
\newcommand{\npdf}[3]{\ensuremath{\frac{\partial^{#3}#1}{\partial#2}}}
%
\newcommand{\gsv}{\vct{\gamma}{s}{}}
\DeclareDocumentCommand{\gammav}{O{} O{}}{\vct{\gamma}{#1}{#2}}
%
\newcommand{\pd}[2]{\ensuremath{#1}_{#2}}
\newcommand{\pda}[3]{\ensuremath{#1}_{#2}\pdp{#3}}
\newcommand{\E}[2]{\ensuremath{\mathbb{E}_{#1}\psp{#2}}}
\newcommand{\cE}[3]{\ensuremath{\mathbb{E}_{#1}\psp{#2\vert#3}}}
\newcommand{\He}[2]{\ensuremath{\mathbb{H}_{#1}\psp{#2}}}
\newcommand{\cHe}[3]{\ensuremath{\mathbb{H}_{#1}\psp{#2\vert#3}}}
\newcommand{\KL}[2]{\ensuremath{\mathbb{D}_{KL}\psp{#1\vert\vert#2}}}
\newcommand{\I}[2]{\ensuremath{\mathbb{I}\psp{#1;#2}}}
\newcommand{\prs}{\ensuremath{\pdp{\Omega,\mathcal{A},\mathit{P}}}}
\newcommand{\prbs}[1]{\ensuremath{\pdp{\Omega_{#1},\mathcal{A}_{#1}}}}
%
\DeclareDocumentCommand{\iO}{O{t}}{
\ensuremath{\int_{\Omega_{#1}}}}

\DeclareDocumentCommand{\xtv}{ O{} O{} O{} }{\vct{\tilde{x}}{#1}{#2}#3}
%
\DeclareDocumentCommand{\xiv}{ O{} O{} O{} }{\vct{\xi}{#1}{#2}#3}
\DeclareDocumentCommand{\xitv}{ O{} O{} O{} }{\vct{\tilde{\xi}}{#1}{#2}#3}
%
\DeclareDocumentCommand{\pxv}{ O{} O{} O{}}{\pdp{\xv[#1][#2][#3]}}
\DeclareDocumentCommand{\pxtv}{ O{} O{} O{}}{\pdp{\xtv[#1][#2][#3]}}
\DeclareDocumentCommand{\pxtuv}{ O{} O{} O{}}{\pdp{\xtv[#1][#2][#3],u\pxtv[#1][#2][#3]}}
%
\DeclareDocumentCommand{\pxiv}{ O{} O{} O{}}{\pdp{\xiv[#1][#2][#3]}}
\DeclareDocumentCommand{\pxitv}{ O{} O{} O{}}{\pdp{\xitv[#1][#2][#3]}}
\DeclareDocumentCommand{\pxituv}{ O{} O{} O{}}{\pdp{\xitv[#1][#2][#3],u\pxitv[#1][#2][#3]}}
%
\DeclareDocumentCommand{\afc}{ O{} O{} O{}}{\vct{a}{}{}\pxtuv[#1][#2][#3]}
\DeclareDocumentCommand{\aftc}{ O{} O{} O{}}{\vct{\tilde{a}}{}{}\pxtuv[#1][#2][#3]}
%
\newcommand{\timedomain}{(0,T)}
\newcommand{\domain}{\Omega}
\newcommand{\FreeSurface}{\Gamma}

\newcommand{\hx}{\ensuremath{\vct{h}{}{}\uavr{x}{}{}}}
\newcommand{\hxi}{\ensuremath{\vct{h}{}{}\uavr{x}{i}{}}}
\DeclareDocumentCommand{\Sset}{O{}}{\ensuremath{\mathcal{S}#1}}
\newcommand{\pjxy}[1][\davvr{x}{}{}{y}{}{}]{\ensuremath{\mathrm{P}_{\scriptscriptstyle{\mathcal{XY}}}#1}}
\DeclareDocumentCommand{\Pr}{O{}}{\ensuremath{\mathit{P}#1}}
\DeclareDocumentCommand{\emean}{O{}}{\ensuremath{\mathbb{E}_{{\scriptscriptstyle{#1}}}}}
\DeclareDocumentCommand{\pxy}{O{}}{\ensuremath{p_{{\scriptscriptstyle{XY}}}#1}}
\DeclareDocumentCommand{\px}{O{}}{\ensuremath{p_{{\scriptscriptstyle{X}}}#1}}
\DeclareDocumentCommand{\qx}{O{}}{\ensuremath{q_{{\scriptscriptstyle{X}}}#1}}
\DeclareDocumentCommand{\qy}{O{}}{\ensuremath{q_{{\scriptscriptstyle{Y}}}#1}}
\DeclareDocumentCommand{\pux}{O{}}{\ensuremath{u_{{\scriptscriptstyle{X}}}#1}}
\DeclareDocumentCommand{\puy}{O{}}{\ensuremath{u_{{\scriptscriptstyle{Y}}}#1}}
\DeclareDocumentCommand{\puxy}{O{}}{\ensuremath{u_{{\scriptscriptstyle{XY}}}#1}}

\DeclareDocumentCommand{\py}{O{}}{\ensuremath{p_{{\scriptscriptstyle{Y}}}#1}}
\DeclareDocumentCommand{\pycx}{O{}}{\ensuremath{p_{{\scriptscriptstyle{Y\vert X}}}#1}}
\DeclareDocumentCommand{\pxcy}{O{}}{\ensuremath{p_{{\scriptscriptstyle{X\vert Y}}}#1}}
\DeclareDocumentCommand{\qycx}{O{}}{\ensuremath{q_{{\scriptscriptstyle{Y\vert X}}}#1}}
\DeclareDocumentCommand{\puycx}{O{}}{\ensuremath{u_{{\scriptscriptstyle{Y\vert X}}}#1}}
\DeclareDocumentCommand{\xsimpd}{O{}}{\ensuremath{\vct{x}{}{}\sim\pdata[]}}
\DeclareDocumentCommand{\Wv}{O{} O{}}{\ensuremath{\vct{W}{#1}{#2}}}

\newcommand{\Sent}{\ensuremath{\mathbb{S}}}
\newcommand{\Sxy}{\ensuremath{\mathbb{S}\pdp{X,Y}}}
\newcommand{\Sxcy}{\ensuremath{\mathbb{S}\pdp{X\vert Y}}}
\newcommand{\Sycx}{\ensuremath{\mathbb{S}\pdp{Y\vert X}}}
\DeclareDocumentCommand{\Sd}{O{} O{}}{\ensuremath{\mathbb{S}\pdp{#1\vert\vert#2}}}
\newcommand{\Spdq}{\Sd[\px][\qx]}
\newcommand{\Sqdp}{\Sd[\qx][\px]}
\newcommand{\Sx}{\ensuremath{\mathbb{S}\pdp{X}}}
\newcommand{\Sy}{\ensuremath{\mathbb{S}\pdp{Y}}}
\DeclareDocumentCommand{\DJS}{O{p} O{q}}{\ensuremath{\mathbb{D}_{JS}\pdp{#1\vert\vert #2}}}
%\newcommand{\xv}{\mathbf{x}}
%\newcommand{\uv}{\mathbf{u}}
%\newcommand{\vv}{\mathbf{v}}
%\newcommand{\Mv}{\mathbf{M}}
%\newcommand{\nv}{\mathbf{n}}
%\newcommand{\Nv}{\mathbf{N}}
\DeclareDocumentCommand{\Km}{ O{} O{}}{\tens{K}{#1}{#2}}
\DeclareDocumentCommand{\Mm}{ O{} O{}}{\tens{M}{#1}{#2}}
\DeclareDocumentCommand{\Wm}{ O{} O{}}{\tens{W}{#1}{#2}}
\DeclareDocumentCommand{\Sm}{ O{} O{}}{\tens{S}{#1}{#2}}
\DeclareDocumentCommand{\Um}{ O{} O{}}{\tens{U}{#1}{#2}}
%\DeclareDocumentCommand{\Fext}{ O{} O{}}{\tens{F}{#1}{#2}}
\DeclareDocumentCommand{\FextF}{ O{} O{}}{\vct{\hat{F}}{#1}{#2}}
\DeclareDocumentCommand{\Fint}{ O{} O{int}}{\vct{F}{#1}{#2}}
\DeclareDocumentCommand{\Fext}{ O{} O{ext}}{\vct{F}{#1}{#2}}
\DeclareDocumentCommand{\diagKm}{ O{} O{}}{\mathsf{diag}\pdp{\Km[#1][#2]}}
\DeclareDocumentCommand{\diagMm}{ O{} O{}}{\mathsf{diag}\pdp{\Mm[#1][#2]}}
\DeclareDocumentCommand{\diagsc}{ O{}}{\mathsf{diag}\pdp{#1}}
%\newcommand{\Ov}{\mathbf{0}}
\newcommand{\density}{\ensuremath{\rho}}
\DeclareDocumentCommand{\Lla}{ O{\pdp{\xv}}}{\ensuremath{\lambda#1}}
\DeclareDocumentCommand{\Lmu}{ O{\pdp{\xv}}}{\ensuremath{\mu#1}}
\DeclareDocumentCommand{\diagsc}{ O{}}{\mathsf{diag}\pdp{#1}}
\newcommand{\vload}{\ensuremath{\vct{f}{v}{}}}
\newcommand{\sload}{\ensuremath{\vct{f}{s}{}}}

\newcommand{\Celas}{\mathbb{C}}
\newcommand{\R}{\mathbb{R}}
\newcommand{\kv}{\mathbf{k}}
\newcommand{\dkv}{\Delta \mathbf{k}}
\newcommand{\Lv}{\mathbf{L}}
\theoremstyle{plain}
\newtheorem*{theorem*}{Theorem}
\flushbottom


% MECHANICS
\newcommand{\gradxU}{\ensuremath{\gradt{u}{x}}}
\newcommand{\gradpU}{\ensuremath{\gradt{u}{p}}}
\newcommand{\gradpUG}{\ensuremath{\gradt{u_G}{p}}}
\newcommand{\gradxUG}{\ensuremath{\gradt{u_G}{x}}}
\newcommand{\gradxUT}{\ensuremath{\gradtT{u}{x}}}

\newcommand{\gradpUT}{\ensuremath{\gradtT{u}{p}}}
\newcommand{\gradxUGT}{\ensuremath{\gradtT{u_G}{x}}}

\newcommand{\gradxUh}{\ensuremath{\tens{\mathbb{D}_{x}u}{}{h}}}
\newcommand{\gradxUhT}{\ensuremath{\tens{\mathbb{D}^T_{x}u}{}{h}}}

\newcommand{\Fgrad}{\ensuremath{\tens{\mathbb{F}}{}{}}}
\newcommand{\FgradT}{\ensuremath{\tens{\mathbb{F}}{}{T}}}
\newcommand{\FgradTinv}{\ensuremath{\tens{\mathbb{F}}{}{-T}}}
\newcommand{\Cstr}{\ensuremath{\tens{\mathbb{C}}{}{}}}
\newcommand{\cartFgrad}{\ensuremath{\sum_{n=1}^{3}\frac{\partial \xv}{\partial p_n}\otimes \vct{i}{n}{}}}
\newcommand{\cylFgrad}{\ensuremath{\frac{\partial \xv}{\partial r}\otimes \vct{i}{r}{}+\frac{\partial \xv}{\partial \theta}\otimes \frac{\vct{i}{\theta}{}}{r} + \frac{\partial \xv}{\partial z}\otimes \vct{i}{z}{}}}
\newcommand{\curvFgrad}{\ensuremath{
\sum_{i}\frac{\partial \xv}{\partial \xi_i}\otimes\gradv{\xi_i}{p}
}}

\DeclareDocumentCommand{\Estrain}{O {} O{}}{\ensuremath{\tens{\mathbb{E}}{#1}{#2}}}
\newcommand{\FEstrain}{\ensuremath{\frac{1}{2}\pdp{\FgradT\Fgrad-\ID}}}
\newcommand{\UEstrain}{\ensuremath{\frac{1}{2}\pdp{
\gradpU+\gradpUT+\gradpUT.\gradpU}}}
\newcommand{\cartEspilon}{ \ensuremath{\sum_{n=1}^{3}\frac{\partial \uv}{\partial x_n}\otimes_S \vct{i}{n}{}} }
\DeclareDocumentCommand{\Sstress}{O {} O{}}{\ensuremath{\tens{\mathbb{S}}{#1}{#2}}}
%
\newcommand{\gradfstress}{\ensuremath{\gradt{f}{\dsigma}{}}}
\newcommand{\gradfhidd}{\ensuremath{\gradh{f}{\dchi}{}}}
\newcommand{\gradgstress}{\ensuremath{\gradt{g}{\dsigma}{}}}
\newcommand{\gradghidd}{\ensuremath{\gradh{g}{\dchi}{}}}
\newcommand{\gradgX}{\ensuremath{\gradt{g}{X}{}}}
\newcommand{\gradfX}{\ensuremath{\gradt{f}{X}{}}}
\newcommand{\gradchieta}{\ensuremath{\tensf{D}{\eta}{}\chih }}

% energies
\newcommand{\EnInt}{\ensuremath{\mathcal{E}_{int}}}
\newcommand{\EnExt}{\ensuremath{\mathcal{E}_{ext}}}
\newcommand{\PInt}{\ensuremath{\mathcal{P}_{int}}}
\newcommand{\PExt}{\ensuremath{\mathcal{P}_{ext}}}
\newcommand{\PKin}{\ensuremath{\mathcal{P}_{kin}}}
\newcommand{\EKin}{\ensuremath{\mathcal{E}_{kin}}}
\newcommand{\EInt}{\ensuremath{\mathcal{E}_{i} }}
\newcommand{\UInt}{\ensuremath{\mathcal{U}}}
\newcommand{\GGibs}{\ensuremath{\mathcal{G}}}
\DeclareDocumentCommand{\Hent}{O{}}{\ensuremath{\mathcal{H}_{#1}}}
\DeclareDocumentCommand{\hessian}{O{} O{}}{\ensuremath{\tens{\mathbb{H}}{#1}{#2}}}
\DeclareDocumentCommand{\MBL}{O{} O{} O{}}{\ensuremath{\mathcal{M}^{#3}\pdp{#1,#2}}}
\DeclareDocumentCommand{\KBL}{O{} O{}}{\ensuremath{\mathcal{K}\pdp{#1,#2}}}
\newcommand{\plm}{\ensuremath{\lambda}}
\newcommand{\plmdot}{\ensuremath{\dot{\lambda}}}
\newcommand{\dplm}{\ensuremath{\delta\lambda}}


\newcommand*\circled[1]{\tikz[baseline=(char.base)]{
		\node[shape=circle,draw,inner sep=2pt] (char) {#1};}}

%\newcommand{\iff}{\ensuremath{\Leftrightarrow}}

%% BEAM
%\newcommand{\vctr}[1]{\ensuremath{\boldsymbol#1}}
%\newcommand{\vctri}[2]{\ensuremath{\boldsymbol{#1_{#2}}}}
%\newcommand{\tenseps}{\ensuremath{\epsilon\hspace{-1.15mm}\epsilon}}
%\newcommand{\tensigma}{\ensuremath{\sigma\hspace{-2mm}\sigma}}
%\newcommand{\tenstau}{\ensuremath{\tau\hspace{-2mm}\tau}}
%\newcommand{\tensOmega}{\ensuremath{\Omega\hspace{-2.35mm}\Omega}}
%\newcommand{\tensomega}{\ensuremath{\omega\hspace{-2.35mm}\omega}}
%\newcommand{\tensnull}{\ensuremath{0\hspace{-1.45mm}0}}
%\newcommand{\tnsr}[1]{\ensuremath{\mathbb{#1}}}
%\newcommand{\tnsri}[2]{\ensuremath{\mathbb{#1}_{\boldsymbol{#2}}}}
%\newcommand{\tnsrq}[1]{\ensuremath{\mathbb{#1}}}
%\newcommand{\prscal}[2]{\ensuremath{\left\langle#1,#2\right\rangle}}

\graphicspath{{\lightimages}{\generalimages}{\oldimages}}
%%%%
\usepackage[utf8]{inputenc}
\usepackage{verbatim}
\usetheme{umu}
\usepackage{amsmath,empheq}
\usepackage{media9}
\usepackage{multimedia}
\usepackage{animate}
%%% Bibliography
\usepackage[style=authoryear,backend=biber]{biblatex}
\addbibresource{\basics/MyCollection.bib}
\setlength{\TPHorizModule}{1mm}
\setlength{\TPVertModule}{1mm}
% Author names in publication list are consistent 
% i.e. name1 surname1, name2 surname2
% See https://tex.stackexchange.com/questions/106914/biblatex-does-not-reverse-the-first-and-last-names-of-the-second-author
\DeclareNameAlias{author}{first-last}

%%% Suppress biblatex annoying warning
\usepackage{silence}
\WarningFilter{biblatex}{Patching footnotes failed}

%%% Some useful commands
% pdf-friendly newline in links
\newcommand{\pdfnewline}{\texorpdfstring{\newline}{ }} 
% Fill the vertical space in a slide (to put text at the bottom)
\newcommand{\framefill}{\vskip0pt plus 1filll}

%%%%%%%%%%%%%%%%%%%%%%%%%%%%%%%%%%%%%%%%%%%%%%%%%%%%%%%%%%%%%%%%%%%%%%%%%%%%%%%%%%%%%
%%%%%%%%%%%%%%%%%%%%%%%%%%%%%%% YOUR PRESENTATION BELOW %%%%%%%%%%%%%%%%%%%%%%%%%%%%%
%%%%%%%%%%%%%%%%%%%%%%%%%%%%%%%%%%%%%%%%%%%%%%%%%%%%%%%%%%%%%%%%%%%%%%%%%%%%%%%%%%%%%
\title[CentraleSupélec]{CNL: hands on [Part 1]}
%\date[\today]{\small\today}
\author[Filippo Gatti]{
	Filippo Gatti, Ph.D.
	\pdfnewline
	\texttt{filippo.gatti@centralesupelec.fr}
	\pdfnewline  Lab.MSSMat UMR CNRS 8579
	\pdfnewline CentraleSupélec
}
\institute{M2 Géo-mécanique : Ouvrages, Eau, Réservoirs (GEO2)}
\usepackage{listings}
\usepackage{color}
\usepackage{relsize}
% Custom colors
\definecolor{dkgreen}{rgb}{0,0.6,0}
\definecolor{gray}{rgb}{0.5,0.5,0.5}
\definecolor{mauve}{rgb}{0.58,0,0.82}
\definecolor{deepblue}{rgb}{0,0,0.5}
\definecolor{deepred}{rgb}{0.6,0,0}
\definecolor{deepgreen}{rgb}{0,0.5,0}
% Default fixed font does not support bold face
\DeclareFixedFont{\ttb}{T1}{txtt}{bx}{n}{12} % for bold
\DeclareFixedFont{\ttm}{T1}{txtt}{m}{n}{12}  % for normal
% Python style for highlighting
\newcommand\pythonstyle{\lstset{
		language=Python,
		basicstyle=\ttm,
		otherkeywords={self},             % Add keywords here
		keywordstyle=\ttb\color{deepblue},
		emph={MyClass,__init__,__class__,__name__},          % Custom highlig´hting
		emphstyle=\ttb\color{deepred},    % Custom highlighting style
		stringstyle=\color{deepgreen},
		frame=tb,                         % Any extra options here
		showstringspaces=false,
		commentstyle=\color{gray},
		tabsize=4
}}
% Java
%\lstset{frame=tb,
%	language=Java,
%	aboveskip=3mm,
%	belowskip=3mm,
%	showstringspaces=false,
%	columns=flexible,
%	basicstyle={\small\ttfamily},
%	numbers=none,
%	numberstyle=\tiny\color{gray},
%	keywordstyle=\color{blue},
%	commentstyle=\color{dkgreen},
%	stringstyle=\color{mauve},
%	breaklines=true,
%	breakatwhitespace=true,
%	tabsize=3
%}


% Python environment
\lstnewenvironment{python}[1][]
{
	\pythonstyle
	\lstset{#1}
}
{}

% Python for external files
\newcommand\pythonexternal[2][]{{
		\pythonstyle
		\lstinputlisting[#1]{#2}}}

% Python for inline
\newcommand\pythoninline[1]{{\pythonstyle\lstinline!#1!}}

\usepackage{minibox}

\begin{document}
	
	\begin{frame}
	\titlepage
\end{frame}

\begin{frame}{Outline}
\tableofcontents
\end{frame}

\section{Élastoplasticité : concepts avancés}
\subsection{Modèles élastoplastiques avancés}
\label{subsec:advanced-elasto-plasticity}
\framecard{Modèles élastoplastiques avancés}


\begin{frame}{Problème élastoplastique}
\txb{125}{0}{11}{
	\begin{itemize}
		\item Ingrédients pour décrire l'évolution élastoplastique?
	\begin{enumerate}
		\item seuil de plasticité
		\item fonction de charge (critère de rupture)
		\item écrouissage (évolution du seuil de plasticité)
		\item écoulement plastique
		\item problème d'optimisation
		\item module d'écrouissage et multiplicateur plastique
		\item tenseur de rigidité élastoplastique
	\end{enumerate}

\end{itemize}
}
\end{frame}

\begin{frame}{Seuil de plasticité}
\txb{123}{2}{11}{
\bsys[][
&f\pscp< 0 & \text{elasticity}\\
&f\pscp= 0 & \text{plasticity}
][plastic-threshold]
\centering
\igwf{elastic-domain}{80mm}
}
\end{frame}

\begin{frame}{Fonction de charge}
\txb{123}{2}{11}{
	\bsys[][
	&f\pscp-\sigma_{yld}\leq 0 & \text{limite élastique}\\
	&f_{ult}\pdp{\stress}-\sigma_{ult}\leq 0, \quad(\text{for} \norm{\strainpl}\to\infty) & \text{surface limite}
	][plastic-loci]
	\centering
\bfc{\subfw{deviator_plane_2}{50mm}{}
\subfw{mises_locus_3D}{50mm}{}
}{Modèle de Armstrong-Frederick \cite{PhD_Gatti_2017}}{deviator_plane_AF}
}
\end{frame}

\begin{frame}{Fonction de charge}
\txb{123}{2}{11}{Modèle de Armstrong-Frederick \vspace{-4mm}
\bsys[][
& f \pscp= \fvm[\pdp{\kbackstress}] - \sigma_{yld}-R\pdp{r}\leq 0 & \text{limite élastique}\\
&f_{ult}\pdp{\stress}=\sqrt{3J_2\pdp{\deviator}} - \sigma_{yld}-R_\infty\leq 0 & \text{surface limite}
][AF-surfaces]
\animategraphics[height=45mm,keepaspectratio,autoplay,loop,poster,loop]{1}{mises_cylinder_}{0}{2}
}
\txb{60}{63}{45}{
Variables d'écrouissage :
\bsys[][
& \chih = \pbp{\backstress,R} & \text{statiques}\\
& \etah = \pbp{\kbackstress,r} & \text{cinématiques}
][AF-hardening-variables]
}
\end{frame}

\begin{frame}{Variables d'écrouissage}
\txb{125}{2}{11}{
	\begin{itemize}
		\item Variables d'écrouissages $\equiv$ variables internes
		\item Utilisées pour décrire l'évolution de $f\pscp$
		\item $\chih$: variables d'écrouissage statiques\\ $\to$ collection des variables scalaires, vectorielles, tensorielles \beq{\chih=\pbp{R,\theta,...,\backstress,\tens{Y}{}{},...}}{chi-static}
		\item $\etah$: variables d'écrouissage cinématiques \\$\to$ collection des variables scalaires, vectorielles, tensorielles
		\item Modèle de Armstrong-Frederick $ \etah=\pbp{\kbackstress,r}$:
			\bsys[\chih\equiv][
		& \backstress\pdp{\kbackstress}=\frac{2}{3}C\kbackstress&\\
		&R\pdp{r} = R_{\infty}\pdp{1- \exp\pdp{-br}} & ][]
	\end{itemize}
}
\txb{38}{88}{62}{
\igwf{hardening-scheme}{40mm}
}
\end{frame}


\begin{frame}{Résolution par étapes}
\txb{123}{2}{11}{
	Évolution continue de $f$ en temps : $f\pdp{t+\delta t}\approx f\pdp{t}+\delta f$
	\bsys[\text{si  } f\pdp{t} \equiv 0:][
	& \delta f < 0 \to f\pdp{t+\delta t} < 0& \text{déchargement élastique}\\
	& \delta f = 0 \to f\pdp{t+\delta t} = 0& \text{chargement plastique}
	][cond_consistency_ep]
	\vspace{-3mm}
	\begin{enumerate}
		\item Hypothèse 0: $\dstrainpl=\tens{0}{}{} \to \dstress[][trial]=\Del : \dstrain$\\
		\item Vérifier la valeur de $f\pdp{\stress+\dstress[][trial];\chih}\leq 0$ 
		\item si $f\pdp{\stress+\dstress[][trial];\chih}> 0\to$ correction: $f\pdp{\stress+\dstress[][];\chih+\dchih}=0$
	\end{enumerate}
	\centering
	\ighf{nl_int_1step}{25mm}
	\ighf{nl_int_1step_unl}{25mm}
}
\end{frame}

\begin{frame}{Écrouissage}
\txb{123}{2}{11}{
	\bsys[][
	& f\pdp{t}=0 \text{ and } \gradfstress : \dstress[][trial] > 0 & \text{écrouissage positif}\\
	& f\pdp{t}=0 \text{ and } \gradfstress : \dstress[][trial] = 0 & \text{plasticité parfaite ($f=f_{ult}$)}\\
	& f\pdp{t+\delta t}=0 \text{ and } \gradfstress : \dstress[][trial] < 0 & \text{écrouissage négatif}\\
	& f\pdp{t+\delta t}<0 \text{ and } \gradfstress : \dstress[][trial] < 0 & \text{écrouissage négatif}
	][hardening]
	\centering
	\ighf{hardening}{32mm}
	\ighf{hardening-vs-perfect}{32mm}
}
\end{frame}

\begin{frame}{Loi d'écoulement}

\txb{95}{28}{11}{
\bsys[][
& \dstrainpl=\dplm.\gradgstress, \quad\detah =- \dplm.\gradghidd & \text{non-associée}\\
& \dstrainpl=\dplm.\gradfstress,\quad  \detah =- \dplm.\gradfhidd & \text{associée} (f=g) 
][flow-rule]
}
\txb{123}{2}{40}{
	\beq{\plm=\int_{0}^{t}\sqrt{\frac{2}{3}\dco{\dstrainpl}}dt=}{plastic_multiplier}
\igwf{flow-rule}{80mm}\igwf{non-associated-fr}{40mm}
}
\end{frame}

\begin{frame}{Loi d'écoulement}
\txb{123}{2}{11}{
	Modèle de Armstrong-Frederick : $\lambda,\mu,C,\kappa,b, R_\infty,\sigma_{yld}$ : paramètres du modèles\\
	Potentiel plastique: \beq{g\pscp=f\pscp+\underbrace{\frac{3}{4}\frac{\kappa}{C}\backstress:\backstress}_{\text{\textbf{fading memory}}} }{AF-pp}
	\beq{\dstrainpl  = \dplm\gradgstress = \dplm\gradfstress=\dplm\frac{3}{2}\frac{\deviator-\backstress}{\fvm}}{AF-depl}
	\bsys[\detah\equiv][
	& \dkbackstress = \dplm\gradgX =\dplm\frac{3}{2}\frac{\deviator-\backstress}{\fvm}-\dplm\frac{3\kappa}{2C}\backstress\\
	& \delta r = -\dplm \frac{\partial g}{\partial R}=-\dplm \frac{\partial f}{\partial R}=\dplm \to r\equiv  \lambda
][AF-flow-rule]
}
\end{frame}

\begin{frame}{Évolution du seuil de plasticité}
\txb{125}{2}{11}{
\beq{\dchih=\gradchieta:\detah}{grad_hidd_var}
\centering
\ighf{linear-kinematic-prager}{32mm}\ighf{nlinear-kinematic-prager}{32mm}\ighf{isotropic-hardening}{32mm}
}
\end{frame}

\begin{frame}{Évolution du seuil de plasticité}
\txb{125}{2}{11}{
Modèle de  Armstrong-Frederick:
\bsys[\dchih\equiv][
& \dbackstress\pdp{\kbackstress}=\frac{2}{3}C\dkbackstress = \pdp{ C\frac{\deviator-\backstress}{\fvm}-\kappa\backstress}\dplm\\
&\delta R\pdp{r} = b\pdp{R_{\infty}-R\pdp{r}} \dplm
][AF-flow-rule]
}
\end{frame}

\begin{frame}{Problème d'optimisation}
\txb{125}{2}{11}{
	Conditions de Karush–Kuhn–Tucker conditions:
\bsys[\mathcal{K}:][
&f\pscp\leq 0, \\
&\plm \geq 0, & \forall f\pscp \in  \mathbb{S} \times \mathbb{K}  \\
&\dplm f\pscp = 0
][KKT_consistency]
$\dplm$ est calculé à partir des conditions de consistance élastoplastique:
$$\text{if } f= 0, \to \delta f = 0 \to f\pdp{t+\delta t} = 0,\quad \text{chargement plastique}$$\vspace{-5mm}
%\beq{\delta f = \gradfstress:\underbrace{\Del:\pdp{\dstrain-\dplm\gradgstress}}_{\dstress}+\gradfhidd:\underbrace{\gradchieta}_{\dchih}:\etah}{cond_consistency_1}
\beqh[\delta f = \gradfstress:\underbrace{\Del:\pdp{\dstrain-\dplm\gradgstress}}_{\dstress}+\gradfhidd:\underbrace{\gradchieta:\gradghidd\dplm}_{\dchih=\gradchieta:\detah}=0][cc][][ABC]
}
\end{frame}

\begin{frame}{Module d'écrouissage}
\txb{125}{2}{11}{
	En résolvant l'\hyperlink{ABC}{\Cref{eq:cc}}, on trouve l'incrément du multiplicateur plastique:
\beq{\delta f = 0\to \dplm=\frac{\gradfstress:\Del:\dstrain}{\underbrace{\gradfstress:\Del:\gradgstress}_{\mathcal{H}_{c}\text{=module d'ecrouissage critique}} - \underbrace{\gradfhidd:\gradchieta:\gradghidd}_{\mathcal{H}\text{=module d'ecrouissage}}}}{plm_sol}
Modèle de Armstrong-Frederick:
}

\txb{60}{2}{50}{
{\scriptsize
\bsys[][
& \gradfX=-\frac{3}{2}\frac{\deviator-\backstress}{\fvm};\frac{\partial f}{\partial R}=-1\\
&  \tens{\delta X}{}{}=\pdp{ C\frac{\deviator-\backstress}{\fvm}-\kappa\backstress}\dplm\\
&\delta R = b\pdp{R_{\infty}-R\pdp{r}} \dplm
][AF-Hcomp]}
}

\txb{60}{62}{55}{
	{\scriptsize
	\bsys[][
	& \Hent = C-\frac{3\kappa}{2}\frac{\pdp{\deviator-\backstress}:\backstress}{\fvm}+b\pdp{R_{\infty}-R}\\
	& \Hent[c] = 3\mu
	][AF-moduleH]
}
}
\end{frame}


\begin{frame}{Tenseur de rigidité élastoplastique}
\txb{125}{2}{11}{
À l'aide de l'\Cref{eq:plm_sol}, l'incrément élastoplastique de l'état de contrainte peut s'écrire sous forme:
\beq{\dstress = \Del:\pdp{\dstrain+\underbrace{\frac{\gradfstress:\Del:\dstrain}{\Hent-\Hent[c]}}_{-\dplm}\gradgstress}}{dstress-ep}
Le tenseur de rigidité élastoplastique est identifiée :
\beq{\Dep=\Del:\pdp{\IDD+\frac{\gradgstress\otimes\Del:\gradfstress}{\Hent-\Hent[c]}}}{Dep}

\beq{\dstress=\Dep:\dstrain}{dstressDepdstrain}

}
\end{frame}

\subsection{Modèles EP pour les argiles}
\label{subsec:ep-models-clays}
\framecard{Modèles EP pour les argiles}
\begin{frame}{Comportement expérimental}
\txb{125}{0}{11}{
	\centering
\igwf{comportement-typique}{125mm}
}
\end{frame}

\begin{frame}{Ingrédients de modélisation}
\txb{125}{2}{11}{
\begin{itemize}
	\item<+-> comportement volumétrique/déviatorique couplés
	\begin{enumerate}
		\item<+->  volumétrique
		\begin{itemize}
			\item compression isotrope
			\item compression oedométrique (conditions à repos, in-situ$\to$ consolidation)
			\item variables intéressées: $\stressm'(=p')$ et $\evol$ (ou index des vides $e$)
		\end{itemize} 
		\item<+->  déviatorique
		\begin{itemize}
			\item comportement sous charge triaxial/multiaxial
			\item variables intéressées: $q-p'$ et $\edev,\evol$ ($\edev=\sqrt{\frac{4}{3}J_2\pdp{\strain}}$)
		\end{itemize}
	\end{enumerate}
	\item<+-> rôle de la pression d'eau interstitielle $u_w$:
	\begin{enumerate}
		\item<+->  conditions drainée: $\stress=\stress[][']+u_w\ID$ (Terzaghi)
		\item<+->  conditions non-drainée: $\devol=0$ (sollicitations dynamiques rapides)
	\end{enumerate}
\item<+-> relation élastoplastique (conditions triaxiales)
\beq{
\triaxdstress=\Deptriax\triaxdstrain \quad  \triaxdstrain=\Ceptriax\triaxdstress 
}{}
\end{itemize}
}
%\bfc{\igwf{example-oedo}{60mm}}{\cite{Book_Bardet_1997_Experimental_soil_mechanics}}{comport-repere}
\end{frame}


\begin{frame}{\hypertarget{frame:comp-iso}{Compression isotrope}}
\bfc{\ighf{compression-iso}{35mm}\ighf{example-iso}{40mm}{}}{\cite{Book_Bardet_1997_Experimental_soil_mechanics}}{example-oedo}
$\evol=3\depsil_1=Tr\pdp{\strain}$ déformation volumique
\end{frame}

\begin{frame}{Compression odeométrique}
\ighf{chemin-oedometrique}{35mm}\\
En élasticité: $\devol=\delta\depsil_1=\delta\depsil_a=\frac{\delta H}{H}=3K\dsigma_1$
\bfc{\ighf{example-oedo}{30mm}\ighf{compression-e-logp-davis}{30mm}}{\cite{Book_Bardet_1997_Experimental_soil_mechanics}}{example-oedo}
\end{frame}

\begin{frame}{Comportement volumétrique des argiles}
\txb{60}{0}{11}{
	\centering
	\ighf{p-e-plane}{28mm}
}
\txb{60}{60}{11}{
	\begin{itemize}
		\item Comportement élastique non linéaire (linéaire en $\log p'$)
\item $\lambda$ et $\kappa$ dans le plan $\ln p'-e $
\item  $C_c$ et $C_s$ dans le plan $\log p'-e $ et $\log \dsigma_v'-e$
\item $ OCR=\frac{\dsigma'_{vc}}{\dsigma'_v}$ ou $OCR=\frac{p'_c}{p'}$ 
	\end{itemize}
}
\txb{123}{2}{40}{
\begin{itemize}
\item<+-> $\devol=-\frac{\delta e}{1+e_0}=-\frac{\delta e}{\Gamma}$ {\scriptsize (compression positive)}
\item<+-> $e_0$ index de vide à repos {\scriptsize(ou conventionnellement, correspondant à $p' $=1 kPa)}
\item<+-> $\devol = \devol[el]+\devol[pl]$, $\delta e = \delta e^{el}+\delta e^{pl}$
\item<+-> Consolidation normale [virgin consolidation]: \\ 
$\to \delta e = -\lambda \delta \pdp{\ln p'}=  -\lambda \frac{\delta p'}{ p'},\quad \nu = \Gamma-\ln\pdp{\frac{p'}{p'_0}} $ {\scriptsize $\pdp{\Gamma=1+e_0}$}
\item<+-> Décharge élastique [swelling line]: $\delta e ^{el} = -\kappa \frac{\delta p'}{p'},\delta e ^{pl}=-\pdp{\lambda-\kappa}\frac{\delta p'}{p'}$\\
 \vspace{-3mm}
$\to\nu=\nu_c-\kappa\ln\pdp{\frac{p'}{p'_c}}$ \\
$e_c=e_{c0}-\lambda\ln\pdp{\frac{p'_c}{p'_{c0}}}$ ou $\nu_c=\nu_{c0}-\lambda\ln\pdp{\frac{p'_c}{p'_{c0}}}$
\end{itemize}
}
\end{frame}

\begin{frame}{\hypertarget{frame:triax-revo}{Chargement Triaxial}}
\only<1>{\ighf{example-triax-rev}{65mm}
	$$\rho=\sqrt{2J_2\pdp{\deviator}}=\sqrt{\frac{2}{3}}q$$}
\only<2>{
	\txb{65}{0}{35}{
	\igwf{options-triax-rev}{65mm}
}
\txb{60}{65}{35}{
	\begin{itemize}
		\item compression triaxiale: $q>0, \dsigma_a>\dsigma_r$
		\item extension triaxiale: $q<0, \dsigma_a<\dsigma_r$
	\end{itemize}
}
\txb{125}{2}{11}{
	Comportement déviatorique dépende de:
	\begin{itemize}
		\item la consolidation (isotrope ou oedometrique) $\to$ argiles
		\item densité $\to$ sables
	\end{itemize}
}
}
\end{frame}

\begin{frame}{Chargement triaxial}
\txb{125}{0}{5}{
\bfc{\subfw{clays-NC}{60mm}{}
\subfw{loose-sands-triax}{60mm}{}
}{ Compression triaxiale monotone : (a) argiles normalement consolidées; (b) sables lâches (\cite{Ziani_Biarez_1990} d'après Lopez-Caballero)}{triax-loose}
}
\end{frame}

\begin{frame}{Ligne d'état critique}
\bfc{\ighf{normal-consolidation-line-3d}{65mm}}{Ligne d'état critique dans l'espace $q-p'-e$}{normal-consolidation-line-3d}
\end{frame}

\begin{frame}{Modèle de Mohr Coulomb}

	\txb{58}{2}{33}{
		\bfc{\igwf{coulomb}{58mm}}{\cite{Han_et_al_2006}}{coulomb}
	}
	\txb{100}{0}{10}{
		\beq{f\pdp{\stress[][']}=\frac{\vert\dsigma_I-\dsigma_{III}\vert}{2}-\frac{\dsigma_I+\dsigma_{III}}{2}\sin\phi'-c'\cos\phi<0}{coulomb}
\beq{g\pdp{\stress[][']}=\vert\dsigma_I-\dsigma_{III}\vert-\pdp{\dsigma_I+\dsigma_{III}}\sin\psi<0}{coulomb}
	\txb{65}{60}{35}{
		\begin{itemize}

			\item Critère de Coulomb:
			\begin{itemize}
				\item $\phi'$ angle de frottement
				\item $c'$ cohésion drainée  apparente 
			\end{itemize}
		\item angle de dilatance $\psi$ ($f\neq g$)
			\item élastoplasticité parfaite (pas d'écrouissage)
		\end{itemize}
	\centering
	\igwf{perfect-ep}{27mm}
	}
}
\end{frame}

\begin{frame}{Modèle de Mohr-Coulomb}
	\txb{60}{2}{11}{
		\igwf{MC-model}{60mm}
	}
\txb{65}{60}{11}{
\beq{f\pdp{\stress[][']}=\vert q\vert -Mp'-N\leq 0}{MC-failure}
\bsys[][
& \dsigma_{I} = \frac{3p'+2q}{3},\dsigma_{III} = \frac{3p'-q}{3}\\
&M = \frac{6\sin\phi'}{3sgn\pdp{q}-\sin\phi'}\\
& N = \frac{6c'\cos\phi'}{3sgn\pdp{q}-\sin\phi'}][MC-pq-properties]
}
\txb{125}{2}{70}{
\beq{g\pdp{\stress[][']}=\vert q \vert- M_gp'}{g-MC}
\beq{M_g=\frac{6\sin\psi}{3sgn\pdp{q}-\sin\psi}}{Mg-MC}
}
\end{frame}


\begin{frame}{Modèle de Mohr-Coulomb en 3D}
\txb{123}{2}{22}{
	
\bfc{\centering\igwf{mohr-coulomb-3d}{40mm}}{\cite{Panteghini_Lagioia_2014}}{mohr-coulomb-3d}
}
\txb{123}{2}{11}{
	\beq{f\pdp{\stress[][']}\frac{\vert q\vert}{3} \pdp{\sqrt{3}\cos\theta-\sin\theta\sin\phi'}-p'\sin\phi'-2c'\cos\phi'\leq 0}{MC-3D}
}
\end{frame}

\begin{frame}{Rôle de la dilatance}
\txb{123}{2}{7}{
	\only<1>{
	\bfc{\igwf{dilatancy-concept}{90mm}}{Essaie de cisaillement direct. (a) Courbes $\tau-u$; (b) Courbes $v-u$; (c) mécanisme d'inter-locking}{dilatancy-concept}
}
	\only<2>{
	\bfc{\subfw{dilatancy-curves-type}{65mm}{}	\\ \subfw{dilatancy-angles-mohr}{65mm}{}
}{Essaie de cisaillement direct: (a) Courbes typiques; (b) loi d'écoulement associée ($\psi=\phi'$) et non-associé ($\psi=\neq\phi'$)}{dilatancy}
}
}
\end{frame}

\begin{frame}{Comportement volumique}
\txb{83}{2}{11}{
Le modèle de Mohr-Coulomb est trop simple:\\
	\begin{itemize}
		\item<+-> Conditions drainées $\to$ comportement volumique:
		\begin{enumerate}
			\item phase élastique : $\dpeff\uparrow \devolel\uparrow$
			\item phase élastoplastique dépende du $OCR$
			\begin{itemize}
				\item contraction (NC ou  $OCR<4$)
				\item dilatance ( $OCR>4$): mesurée comme $\frac{\devolpl}{\dedevpl}$
			\end{itemize}
		\end{enumerate}
	
		\item<+-> Conditions non-drainées $\devol=0$:
		\begin{enumerate}
			\item phase élastique : 
			\begin{itemize}
				\item 	$\devol=\devolel=0=\Celpp\dpeff \pdp{\Celpq=0} \to \delta p=\duw\uparrow $
			\end{itemize}
			\item phase élastoplastique : 
			\begin{itemize}
				\item $OCR<4\to \duw>0$
				\item $OCR>4\to \duw >0 $ et $\duw<0$
				\item $\dpeff = \Ceppp \cancelto{0}{\devol}+\Ceppq\dedev=\delta p+\duw$
				\item $\dq = \Cepqp \cancelto{0}{\devol}+\Cepqq\dedev$
			\end{itemize}
			\end{enumerate}
	\end{itemize}
}
\txb{40}{85}{20}{
	\igwf{normal-consolidation-line-3d}{40mm}
	%\igwf{cap-model}{40mm}
}
\end{frame}

\begin{frame}{Compression isotrope}
	\txb{125}{2}{11}{
	\begin{itemize}
		\item Comportement élastique : \beq{\frac{\dpeff}{p '}=\frac{1+e}{\kappa}\devolel=\frac{\nu}{\kappa}\devolel}{dp-devolel-iso}
		\item Comportement élastoplastique : \beq{\frac{\dpeff}{p '}=\frac{1+e}{\lambda-\kappa}\devolpl=\frac{\nu}{\lambda-\kappa}\devolpl}{dp-devolpl-iso}
	\end{itemize}
}

\txb{40}{85}{50}{
	\igwf{normal-consolidation-line-3d}{40mm}
	%\igwf{cap-model}{40mm}
}
\end{frame}

\begin{frame}{Rôle de la dilatance}
\txb{125}{2}{11}{
	Incrément de travail élastoplastique (constante le long l'essai): \beq{\delta W^{ep}=\stress[][']:\dstrainpl=p'\devolpl+q\dedevpl}{dWep}
	À l'état critique en compression (argile normalement consolidée):
	\bsys[][
	& \devol\approx\devolpl=0, \quad \dq=M\dpeff\\
	& \delta W^{ep} = M p' \dedevpl=p'\devolpl+q\dedevpl
	][CS-conditions]
}
\txb{75}{0}{55}{
	\bsys[\frac{\devolpl}{\dedevpl}\equiv][
	& M-\frac{q}{p'} > 0 & \text{compression}\\
	& M-\frac{q}{p'} = 0 & \text{CS}\\
	& M-\frac{q}{p'} < 0 & \text{dilatance}
	][conditions-dilatancy-CS]
}
\txb{40}{85}{50}{
	\igwf{normal-consolidation-line-3d}{40mm}
	%\igwf{cap-model}{40mm}
}
\end{frame}

\begin{frame}{Cam-Clay (CCM) et version modifiée(MCCM)}
\txb{125}{2}{11}{
\begin{itemize}
	\item Fonction de charge: 
	\bsys[f\pdp{\stress}\equiv][
	& q+Mp'\ln \pdp{\frac{p'}{p'_c}}\leq 0 & \text{CCM}\\
	& \frac{q^2}{p'^2}+M ^2\pdp{1-\frac{p'_c}{p'}}\leq 0 & \text{MCCM}
	][CCM-failure]
\end{itemize}
}
\txb{50}{0}{50}{
\igwf{CCM-failure}{50mm}
}
\txb{75}{50}{55}{
$p'_c$ : pression de pré-consolidation 
\begin{itemize}
	\item[$\to$] contrôle la taille de la surface 
	\item[$\to$] paramètre d'écrouissage (voir \Cref{eq:dp-devolpl-iso}) \beq{p'_c = p'_{c0}\exp\pdp{\frac{\nu}{\lambda-\kappa}\evolpl}}{pc-evol}
\end{itemize}
}
\end{frame}



%
%\section{Colors}
%
%\begin{frame}{Colors}
%
%For this template we defined four colors, following the graphic profile of Umeå University:
%\begin{itemize}
%\item \textcolor{white}{\marker[UmUBlue]{\texttt{UmUBlue}}}
%\item \textcolor{white}{\marker[UmUGreen]{\texttt{UmUGreen}}}
%\item \textcolor{white}{\marker[UmUPink]{\texttt{UmUPink}}}
%\item \textcolor{white}{\marker[UmUGold]{\texttt{UmUGold}}}
%\end{itemize}
%
%\vskip 0.5cm
%
%You can use these colors as you want in your presentation. For example, you can \textbf{\textcolor{UmUGold}{color the text in gold}} by writing \texttt{\textbackslash\{UmUGold\}\{my gold text\}}.
%
%\vskip 0.5cm
%
%We also redefined many of the most common \LaTex{} and Beamer commands, like \texttt{itemize}, \texttt{block}, etc. You will see samples of these commands in the following slides.
%
%\end{frame}
%
%\section{Blocks}
%
%\begin{frame} 
%\frametitle{This is a page with a title and a subtitle} 
%\framesubtitle{And also some blocks.} 
%\begin{block}{Goal of the mission}
%Shoot in the Death Star's exhaust port and destroy it before it can fire on the Rebel base.
%\end{block} 
%\begin{alertblock}{Take care!}
%TIE Fighters may chase you while approaching the target.
%\end{alertblock} 
%\begin{exampleblock}{Use the force you must}
%Remember your training with Obi-Wan, and use the Force to make the perfect shot.
%\end{exampleblock} 
%
%\end{frame}
%
%
%
%\section{Enumerates, itemizes and description}
%
%\subsection{Enumerates and itemizes}
%
%\begin{frame}{Enumerates and itemizes}
%
%This is an example of \texttt{itemize}.
%\begin{itemize}
%	\item A long time ago in a galaxy far, far away...
%\end{itemize}
%And this is an example of \texttt{enumerate}.
%
%\begin{enumerate} 
%  \item Go to the Death Star.
%  \item Find the exhaust port.
%  \item Make the perfect shot.
%  \item Become an hero.
%\end{enumerate}
%\end{frame}
%
%\subsection{Description}
%
%\begin{frame}[fragile]
%\frametitle{Description}
%This is an example of \texttt{description}.
%
%\begin{description}
%\item<2->[Vader] \emph{I am} your father.
%\item<1->[Luke] No. No! That's not true! \textbf{That's impossible!}
%\end{description}
%
%\begin{uncoverenv}<3>
%  \vskip 0.5cm
%  And while we're here, let's have a look to \texttt{verbatim} as well, to see how we made items appear in arbitrary order:
%  \vskip 0.5cm
%  \begin{verbatim}
%\begin{description}
%  \item<2->[This is the first item] one
%  \item<1->[This is the second item] two
%\end{description}
%  \end{verbatim}
%\end{uncoverenv}
%
%\end{frame}
%
%\section{Maths}
%
%\begin{frame}{Maths}
%A formula will look like this: 
%\begin{center}
% $x^2 + y^2 = z^2$
%\end{center}
%
%You can number equations as well:
%\begin{equation}
%1+1=2
%\end{equation}
%
%\begin{equation}
%1+1=2 \tag{custom label!}
%\end{equation}
%
%\vskip 0.5cm
%
%If you want to use the default \LaTex{} math fonts, just go to \texttt{beamerfontthemeumu.sty} and uncomment the line containing `\texttt{\textbackslash usefonttheme[onlymath]\{serif\}}'.
%
%\end{frame}
%
%\begin{frame}{Theorems}
%
%The usual \texttt{theorem}, \texttt{corollary}, \texttt{definition}, \texttt{definitions}, \texttt{fact}, \texttt{example} and \texttt{examples} blocks are available as well.
%
%\begin{theorem}
%There exists an infinite set.
%\end{theorem}
%\begin{proof}
%This follows from the axiom of infinity.
%\end{proof}
%\begin{example}[Natural Numbers]
%The set of natural numbers is infinite.
%\end{example}
%
%\end{frame}
%
%\section{Other blocks}
%
%\begin{frame}{Other blocks}
%
%Here we display examples of \texttt{abstract}, \texttt{verse}, \texttt{quotation}, and \texttt{quote}.
%
%\vskip 0.5cm
%
%\begin{abstract}
%This is an abstract.
%\end{abstract}
%\begin{verse}
%This is a verse.
%\end{verse}
%\begin{quotation}
%This is a quotation.
%
%\raggedleft -Han Solo
%\end{quotation}
%\begin{quote}
%A quote this is.
%
%\raggedleft -Yoda
%\end{quote}
%
%\end{frame}
%
%\section{Bibliography and Publications}
%\begin{frame}[fragile]

%
%You can cite an article
%\begin{itemize}
%\item normally using \texttt{\textbackslash cite}, e.g.: (\cite{article1})
%\item or display the full citation using \texttt{\textbackslash fullcite}, e.g.:  \fullcite{article1} \\\vspace{0.4cm}
%
%\textit{(n.d.) stands for "no date". \texttt{year=\{A long time ago...\}} is not a date that can be specified in \texttt{bibliography} anyway.}
%\end{itemize}
%
%\vskip 0.5cm
%Look at the code of the following slide to see how to automatically split the bibliography on many slides. You can also use \texttt{\textbackslash nocite\{*\}} to display the non-cited publications as well.
%
%\end{frame}
%\section{Deep-Convolutional Generative Auto-Encoders}
%\framecard{Deep-Convolutional Generative Auto-Encoders}
%\begin{frame}{DC-GAAE}
%\textit{Sample set} $\mathcal{S}:=\pbp{\pdp{\vct{x}{d}{\pdp{i}},\vct{x}{d}{\pdp{i}}}, i\in\psp{m}}\subseteq\mathbb{R}^{3 n_t}\times\mathbb{R}^{3 n_t}$
%	\begin{block}{AE: Auto-Encoder}
%		\begin{itemize}
%			\item[+] Encoding: $$ \vct{z}{}{} = \tens{Q}{}{}\pdp{\vct{x}{d}{}\vert\vct{\theta}{Q}{}}$$ 
%			\item[+] Decoding: $$ \vct{x}{r}{} = \tens{P}{}{}\pdp{\vct{z}{}{}\vert\vct{\theta}{P}{}}$$ 
%			\item[+] Loss: $$\ell_{MSE}\davvr{x}{r}{}{x}{d}{}=\Vert \vct{x}{d}{}-\vct{x}{r}{}\Vert^2$$
%		\end{itemize}
%	\end{block}
%\end{frame}
%
%\begin{frame}{DC-GAAE}
%\begin{block}{AAE: Adversial AutoEncoder}
%	\begin{itemize}
%		\item[+] Encoding: $$ \vct{z}{Q}{} = Q_{\theta_Q}\uavr{x}{d}{}$$ 
%		\item[+] Decoding: $$ \vct{x}{r}{} = P_{\theta_P}\uavr{z}{Q}{}$$ 
%		\item[+] Loss: $$\ell\davvr{x}{r}{}{x}{d}{}=\Vert \vct{x}{d}{}-\vct{x}{r}{}\Vert^2$$
%	\end{itemize}
%\end{block}
%\end{frame}

\section{Bibliography}
\framecard{Bibliography}

\begin{frame}[t,allowframebreaks]
\frametitle{Bibliography}

%\nocite{*} % will display the non-cited publications as well. Useful for a publication list.

\printbibliography

\end{frame}

%\section{Bonus Commands}
%
%\begin{frame}[fragile]
%\frametitle{Framecard}
%
%You can display a frame with a colored background and a huge text in the center using the command \texttt{\textbackslash framecard}.
%\vskip 0.5cm 
%For example, you can write:
%\begin{verbatim}
%\framecard{A SECTION\\TITLE}
%\end{verbatim}
%
%This will display a frame with a orange background and the phrase "A SECTION TITTLE" in the center. You can also use a custom color with \texttt{\textbackslash framecard}:
%\begin{verbatim}
%\framecard{A SECTION\\TITLE}
%\framecard[UmUGreen]{A SECTION TITLE\\
%WITH A CUSTOM COLOR}
%\end{verbatim}
%You can see the results of the commands above in the following slides.
%
%\end{frame}
%
%\framecard{A SECTION \\\vspace{3pt} TITLE}
%\framecard[UmUGreen]{A SECTION TITLE\\\vspace{3pt}WITH A CUSTOM COLOR}
%
%\begin{frame}[fragile]
%\frametitle{Framepic}
%
%You can display a frame with a background image using the command \texttt{\textbackslash framepic}. The image will be \textbf{adapted vertically} to fit the the frame. 
%
%For example, you can write:
%\begin{verbatim}
%\framepic{graphics/darth}{
%	\framefill
%    \textcolor{white}{Luke,\\I am your supervisor}
%    \vskip 0.5cm
%}
%\end{verbatim}
%
%Alternatively, to make the background 50\% transparent, you can write \texttt{\textbackslash framepic[0.5]\{graphics/darth\}...}
%
%
%You can see the results of the commands above in the following slides.
%
%\end{frame}
%
%
%\framepic{graphics/darth}{
%	\framefill
%    \textcolor{white}{Luke,\\I am your supervisor}
%    \vskip 0.5cm
%}
%
%\framepic[0.5]{graphics/darth}{
%	\vfill
%    \begin{flushright}
%    \textcolor{red}{\textbf{Right-aligned text with\\Semi-transparent background}}
%    \end{flushright}	
%}


%\begin{frame}[t,fragile,allowframebreaks]
%\frametitle{Other bonus commands}
%
%We provide two other bonus commands:
%\begin{description}
%\item[\texttt{pdfnewline}] you can use \texttt{\textbackslash pdfnewline} to avoid the annoying \texttt{hyperref} related warnings when using newlines in the document's title, author, etc. For example, in this presentation the author is defined as:
%\begin{verbatim}
%\author[Luke Skywalker]{
%  Luke Skywalker, Ph.D.
%  \pdfnewline
%  \texttt{luke.skywalker@uniud.it}
%}
%\end{verbatim}
%\item[\texttt{marker}] you can use \texttt{\textbackslash marker} to highlight some text. The default color is \marker{pink}, but you can also \marker[UmUGold]{use a custom color}. For example:
%\begin{verbatim}
%\marker{Default color}
%\marker[UmUGold]{Custom Color}
%\end{verbatim}
%\item[\texttt{framefill}] you can use \texttt{\textbackslash framefill} to put the text at the bottom of a slide by filling all the vertical space.
%\end{description}
%
%\end{frame}

\end{document}