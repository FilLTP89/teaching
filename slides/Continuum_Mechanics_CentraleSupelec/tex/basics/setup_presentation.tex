%%==================================================
%% POLIMI THEME
%%==================================================
%\usetheme{polimi_new}
%%==================================================
%% CAMBRIDGE THEME
%%==================================================
\usetheme{CambridgeUS}
\usecolortheme{spruce}
%%==================================================
%\usepackage[applemac]{inputenc}
%\usepackage[T1]{fontenc}
\usepackage[utf8]{inputenc}
\usepackage[francais]{babel}
\usepackage[outdir=\epstopdfimages]{epstopdf}

\usepackage{graphicx}
\usepackage{etex}
\usepackage{listings}
\lstloadlanguages{Matlab}
\lstset{
  basicstyle=\scriptsize\upshape\ttfamily,
  keywordstyle=\color{blue}\bfseries,%keywordstyle=\color{red}
  showstringspaces=false,morekeywords={tha,thv,thd,AT5,AI5,Ain,pga,pgv,pgd,rsd,psa,fsa},basicstyle=\tiny,
}
\usepackage{natbib}
\usepackage{pxfonts}
\usepackage{subfig}
\usepackage{fancyhdr}
\usepackage{fancybox}
\usepackage{hyperref}
\usepackage{lmodern}
\usepackage{adjustbox}
\usepackage{pstricks,pst-node,pst-coil,pst-plot,pst-text,pst-3dplot,pst-3d,pst-grad,pstricks-add}
\usepackage{pst-tree}
\usepackage{hyperref}
\usepackage{relsize}
\usepackage{appendixnumberbeamer}
\usepackage{amsmath,empheq}
\usepackage{pgfgantt}
\usepackage{multimedia}
\usepackage{color}
\usepackage[normalem]{ulem} % either use this (simple) or
\usepackage{soul} % use this (many fancier options)
\usepackage[absolute,overlay]{textpos}
\setlength{\TPHorizModule}{1mm}
\setlength{\TPVertModule}{1mm}
\usepackage{url}
\usepackage{natbib}
\usepackage{animate}
\usepackage{layout}
%% Define a new 'leo' style for the package that will use a smaller font.
\makeatletter
\def\url@leostyle{%
	\@ifundefined{selectfont}{\def\UrlFont{\sf}}{\def\UrlFont{\small}}}
\makeatother
%% Now actually use the newly defined style.
\urlstyle{leo}
%===============================================================
% NEW COLORS
%===============================================================
\definecolor{olive}{rgb}{0.3, 0.4, .1}
\definecolor{fore}{RGB}{249,242,215}
\definecolor{back}{RGB}{51,51,51}
\definecolor{title}{RGB}{255,0,90}
\definecolor{dgreen}{rgb}{0.,0.6,0.}
\definecolor{fgreen}{RGB}{0 212 85}
\definecolor{gold}{rgb}{1.,0.84,0.}
\definecolor{junglegreen}{cmyk}{0.99,0,0.52,0}
\definecolor{bluegreen}{cmyk}{0.85,0,0.33,0}
\definecolor{rawsienna}{cmyk}{0,0.72,1,0.45}
\definecolor{magenta}{cmyk}{0,1,0,0}
\definecolor{POLIMIgray}{RGB}{114,143,165}
\definecolor{IntenseBlue}{rgb}{0,0.7109,1.0000}
\definecolor{IntenseOrange}{rgb}{1.0000,0.4141,0.1875}
\definecolor{IntenseGreen}{rgb}{0.1211,0.7109,0.1445}
\definecolor{sinapsGreen}{RGB}{130,202,63}
\definecolor{sinapsDarkGreen}{RGB}{0,128,0}
\definecolor{sinapsLightGreen}{RGB}{208,237,179}
%====================================================
% TIKZ
%====================================================
\usepackage{tikz}
\usetikzlibrary{calc}
\def\checkmark{\tikz\fill[scale=0.4](0,.35) -- (.25,0) -- (1,.7) -- (.25,.15) -- cycle;}
\tikzstyle{snofill}=[rectangle,draw,minimum size=1.4em,rounded corners]
\tikzstyle{sfill}=[rectangle,draw,fill=gold!20,minimum size=1.4em,rounded corners]
% Defines a `datastore' shape for use in DFDs.  This inherits from a
% rectangle and only draws two horizontal lines.
\makeatletter
\pgfdeclareshape{datastore}{
  \inheritsavedanchors[from=rectangle]
  \inheritanchorborder[from=rectangle]
  \inheritanchor[from=rectangle]{center}
  \inheritanchor[from=rectangle]{base}
  \inheritanchor[from=rectangle]{north}
  \inheritanchor[from=rectangle]{north east}
  \inheritanchor[from=rectangle]{east}
  \inheritanchor[from=rectangle]{south east}
  \inheritanchor[from=rectangle]{south}
  \inheritanchor[from=rectangle]{south west}
  \inheritanchor[from=rectangle]{west}
  \inheritanchor[from=rectangle]{north west}
  \backgroundpath{
    %  store lower right in xa/ya and upper right in xb/yb
    \southwest \pgf@xa=\pgf@x \pgf@ya=\pgf@y
    \northeast \pgf@xb=\pgf@x \pgf@yb=\pgf@y
    \pgfpathmoveto{\pgfpoint{\pgf@xa}{\pgf@ya}}
    \pgfpathlineto{\pgfpoint{\pgf@xb}{\pgf@ya}}
    \pgfpathmoveto{\pgfpoint{\pgf@xa}{\pgf@yb}}
    \pgfpathlineto{\pgfpoint{\pgf@xb}{\pgf@yb}}
 }
}
\makeatother
\usetikzlibrary{arrows}
%====================================================
% TITLE SETTINGS
%====================================================
%\setbeamercolor{frametitle}{fg=blue}
%% MOVE FRAME TITLE
%\setbeamertemplate{frametitle}[default][left,leftskip=2.01cm]
%% ADD LOGO IN FRAME TITLE
%\addtobeamertemplate{frametitle}{}{%
%	\begin{textblock}{20}(1,5)
%		\includegraphics[width=2cm,keepaspectratio]{logo_sinaps}
%\end{textblock}
%}
\usepackage{ragged2e} % besseren Umbruch
\usepackage{booktabs} % spezielle Tabellen zulassen
\usepackage{textpos}
% OTHER SETTINGSS
%\setbeamertemplate{blocks}[rounded=off,shadow=false]
%\setbeamercolor{block title}{bg=sinapsLightGreen!30,fg=sinapsDarkGreen}
%\setbeamercolor{block title}{bg=white,fg=IntenseBlue}
%\setbeamercolor*{palette primary}{use=structure,fg=sinapsDarkGreen,bg=sinapsGreen}
%\setbeamercolor*{palette primary}{use=structure,fg=blue,bg=blue!20}
%\setbeamercolor{palette tertiary}{use=structure,fg=black,bg=blue} % changed this
%%==================================================
%%==================================================
%% COMMON STYLE
%%==================================================
\setbeamertemplate{itemize item}{\scriptsize\raise1.25pt\hbox{\donotcoloroutermaths$\blacktriangleright$}}
\setbeamertemplate{itemize subitem}{\tiny\raise1.5pt\hbox{\donotcoloroutermaths$\blacktriangleright$}}
\setbeamertemplate{itemize subsubitem}{\tiny\raise1.5pt\hbox{\donotcoloroutermaths$\blacktriangleright$}}
\setbeamertemplate{enumerate item}{\insertenumlabel.}
\setbeamertemplate{enumerate subitem}{\insertenumlabel.\insertsubenumlabel}
\setbeamertemplate{enumerate subsubitem}{\insertenumlabel.\insertsubenumlabel.\insertsubsubenumlabel}
\setbeamertemplate{enumerate mini template}{\insertenumlabel}
\setbeamertemplate{footline}[frame number]
%\setbeamertemplate{footline}{ }
\setbeamertemplate{blocks}[rounded][shadow=true]
\setbeamertemplate{navigation symbols}{}
\setbeamertemplate{caption}[numbered]
\setbeamerfont{caption}{size=\scriptsize}
\setbeamertemplate{bibliography item}{}
\setbeamercolor*{bibliography}{fg=black}
\setbeamerfont{section number projected}{%
	family=\rmfamily,series=\bfseries,size=\normalsize}
\setbeamercolor{section number projected}{bg=IntenseGreen,fg=white}
%\setbeamercolor{section number projected}{bg=IntenseBlue,fg=white}



\AtBeginSection[] {
  \begin{frame}<beamer>{Outline}
    \tableofcontents[currentsection,sectionstyle=show/shaded,subsectionstyle=hide]
  \end{frame}
  \addtocounter{framenumber}{-1}
}

\AtBeginSubsection[] {
	\begin{frame}<beamer>{Outline}
		\tableofcontents[currentsection,currentsubsection,subsectionstyle=show/shaded/hide]
	\end{frame}
	\addtocounter{framenumber}{-1}
}

%\def\colorize<#1>{%
%\temporal<#1>{\color{red!5}}{\color{black}}{\color{black}}}
%\def\colorizered<#1>{%
%\temporal<#1>{\color{green}}{\color{red}}{\color{red}}}

%% FOOTNOTES
% LATEX
%\renewcommand*{\thefootnote}{\fnsymbol{footnote}}
%\renewcommand*{\thefootnote}{\arabic{footnote}}
%\setcounter{footnote}{0}
% BEAMER
\makeatother
\renewcommand{\thefootnote}{\ifcase\value{footnote}\or(c)\or
	(\$\$)\or(\$\$\$)\or(\$\$\$\$)\or(\#)\or(\#\#)\or(\#\#\#)\or(\#\#\#\#)\or(\#\#\#\#\#)\fi}
\makeatletter