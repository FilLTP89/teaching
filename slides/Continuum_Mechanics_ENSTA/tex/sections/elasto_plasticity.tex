\subsection{Élastoplasticité}
\label{subsec:elasto-plasticity}
\framecard{Élastoplasticité}
\begin{frame}{Comportement post-élastique}
	\begin{itemize}
		\item le sol entre en rupture si soumit à une déformation déviatorique relevant ou à traction 
		\item normalement la réponse du sol est non linéaire et irréversible 
		\item le sol peut avoir des déformations volumiques importants quand les
		déformations déviatoriques augmentent
		\item le comportement du sol est dépendant de son histoire
	\end{itemize}

\end{frame}

\begin{frame}{Problème élastoplastique}
\txb{125}{0}{11}{
	\begin{itemize}
		\item Ingrédients pour décrire l'évolution élastoplastique?
	\begin{enumerate}
		\item<only@2> Seuil de plasticité
		\item<only@3>  Fonction de charge (critère de rupture)
		\item<only@4-5> écrouissage (évolution du seuil de plasticité)
		\item<only@6> écoulement plastique
		\item<only@7> problème d'optimisation
		\item<only@8> module d'écrouissage et multiplicateur plastique
	\end{enumerate}

\end{itemize}
}
\txb{123}{2}{20}{
	
\only<2>{
\bsys[][
&f\pscp< 0 & \text{elasticity}\\
&f\pscp= 0 & \text{plasticity}
][plastic-threshold]
\centering
\igwf{elastic-domain}{80mm}
}

\only<3>{
	\bsys[][
	&f\pscp-\sigma_{yld}\leq 0 & \text{limite élastique}\\
	&f_{ult}\pscp-\sigma_{ult}\leq 0, \quad(\text{for} \norm{\strainpl}\to\infty) & \text{surface limite}
	][plastic-loci]
	\centering
\igwf{deviator_plane_2}{80mm}
}

\only<4>{
	\vspace{2mm}
	Évolution continue de $f$ en temps : $f\pdp{t+\delta t}\approx f\pdp{t}+\delta f$
	\bsys[\text{si  } f\pdp{t} \equiv 0:][
& \delta f < 0 \to f\pdp{t+\delta t} < 0& \text{déchargement élastique}\\
& \delta f = 0 \to f\pdp{t+\delta t} = 0& \text{chargement plastique}
][cond_consistency_ep]
\vspace{-3mm}
\begin{enumerate}
	\item Hypothèse 0: $\dstrainpl=\tens{0}{}{} \to \dstress[][trial]=\Del : \dstrain$\\
	\item Vérifier la valeur de $f\pdp{\stress+\dstress[][trial];\chih}\leq 0$ 
	\item si $f\pdp{\stress+\dstress[][trial];\chih}> 0\to$ correction: $f\pdp{\stress+\dstress[][];\chih+\dchih}=0$
\end{enumerate}
\centering
\ighf{nl_int_1step}{22mm}
\ighf{nl_int_1step_unl}{22mm}
}
\only<5>{
	\bsys[][
& f\pdp{t}=0 \text{ and } \gradfstress : \dstress[][trial] > 0 & \text{écrouissage positif}\\
& f\pdp{t}=0 \text{ and } \gradfstress : \dstress[][trial] = 0 & \text{plasticité parfaite ($f=f_{ult}$)}\\
& f\pdp{t+\delta t}=0 \text{ and } \gradfstress : \dstress[][trial] < 0 & \text{écrouissage négatif}\\
& f\pdp{t+\delta t}<0 \text{ and } \gradfstress : \dstress[][trial] < 0 & \text{écrouissage négatif}
][hardening]
\centering
\ighf{hardening}{32mm}
\ighf{hardening-vs-perfect}{32mm}
}

\only<6>{
\bsys[][
& \dstrainpl=\dplm.\gradgstress & \text{non-associée}\\
& \dstrainpl=\dplm.\gradfstress, & \text{associée} (f=g) 
][flow-rule]
\beq{\plm=\int_{0}^{t}\sqrt{\frac{2}{3}\dco{\dstrainpl}}dt=}{plastic_multiplier}
\centering
\ighf{flow-rule}{42mm}
}

\only<7>{
	\vspace{2mm}
	Conditions de Karush–Kuhn–Tucker conditions:
\bsys[\mathcal{K}:][
&f\pscp\leq 0, \\
&\plm \geq 0, & \forall f\pscp \in  \mathbb{S} \times \mathbb{K}  \\
&\dplm f\pscp = 0
][KKT_consistency]
$\dplm$ est calculé à partir des conditions de consistance élastoplastique:
$$\text{if } f= 0, \to \delta f = 0 \to f\pdp{t+\delta t} = 0,\quad \text{chargement plastique}$$\vspace{-5mm}
%\beq{\delta f = \gradfstress:\underbrace{\Del:\pdp{\dstrain-\dplm\gradgstress}}_{\dstress}+\gradfhidd:\underbrace{\gradchieta}_{\dchih}:\etah}{cond_consistency_1}
\beqh[\delta f = \gradfstress:\underbrace{\Del:\pdp{\dstrain-\dplm\gradgstress}}_{\dstress}+\gradfhidd:\dchih=0][cc][<7>][ABC]
}
\only<8>{
	\vspace{2mm}
	Loi d'écoulement des variables statiques d'écrouissage $\chih$ :
	\beq{\chih=\pbp{\alpha,\beta,\gamma,...,\tens{\alpha}{}{},\tens{\beta}{}{},\tens{\gamma}{}{},...}\to \dchih=\gradchieta:\detah=\dplm\gradchieta:\gradghidd}{grad_hidd_var}
	$\etah$: variables cinématiques d'écrouissage $\to \detah=\dplm\gradghidd$.\\
	En résolvant l'\hyperlink{ABC}{\Cref{eq:cc}}, on trouve l'incrément du multiplicateur plastique:
\beq{\delta f = 0\to \dplm=\frac{\gradfstress:\Del:\dstrain}{\underbrace{\gradfstress:\Del:\gradgstress}_{\mathcal{H}_{c}\text{=module d'ecrouissage critique}} - \underbrace{\gradfhidd:\gradchieta:\gradghidd}_{\mathcal{H}\text{=module d'ecrouissage}}}}{plm_sol}
}
}
\end{frame}