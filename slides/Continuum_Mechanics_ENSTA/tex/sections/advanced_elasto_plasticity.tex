\subsection{Modèles élastoplastiques avancés}
\label{subsec:advanced-elasto-plasticity}
\framecard{Modèles élastoplastiques avancés}


\begin{frame}{Problème élastoplastique}
\txb{125}{0}{11}{
	\begin{itemize}
		\item Ingrédients pour décrire l'évolution élastoplastique?
	\begin{enumerate}
		\item seuil de plasticité
		\item fonction de charge (critère de rupture)
		\item écrouissage (évolution du seuil de plasticité)
		\item écoulement plastique
		\item problème d'optimisation
		\item module d'écrouissage et multiplicateur plastique
		\item tenseur de rigidité élastoplastique
	\end{enumerate}

\end{itemize}
}
\end{frame}

\begin{frame}{Seuil de plasticité}
\txb{123}{2}{11}{
\bsys[][
&f\pscp< 0 & \text{elasticity}\\
&f\pscp= 0 & \text{plasticity}
][plastic-threshold]
\centering
\igwf{elastic-domain}{80mm}
}
\end{frame}

\begin{frame}{Fonction de charge}
\txb{123}{2}{11}{
	\bsys[][
	&f\pscp-\sigma_{yld}\leq 0 & \text{limite élastique}\\
	&f_{ult}\pdp{\stress}-\sigma_{ult}\leq 0, \quad(\text{for} \norm{\strainpl}\to\infty) & \text{surface limite}
	][plastic-loci]
	\centering
\bfc{\subfw{deviator_plane_2}{50mm}{}
\subfw{mises_locus_3D}{50mm}{}
}{Modèle de Armstrong-Frederick \cite{PhD_Gatti_2017}}{deviator_plane_AF}
}
\end{frame}

\begin{frame}{Fonction de charge}
\txb{123}{2}{11}{Modèle de Armstrong-Frederick \vspace{-4mm}
\bsys[][
& f \pscp= \fvm[\pdp{\kbackstress}] - \sigma_{yld}-R\pdp{r}\leq 0 & \text{limite élastique}\\
&f_{ult}\pdp{\stress}=\sqrt{3J_2\pdp{\deviator}} - \sigma_{yld}-R_\infty\leq 0 & \text{surface limite}
][AF-surfaces]
\animategraphics[height=45mm,keepaspectratio,autoplay,loop,poster,loop]{1}{mises_cylinder_}{0}{2}
}
\txb{60}{63}{45}{
Variables d'écrouissage :
\bsys[][
& \chih = \pbp{\backstress,R} & \text{statiques}\\
& \etah = \pbp{\kbackstress,r} & \text{cinématiques}
][AF-hardening-variables]
}
\end{frame}

\begin{frame}{Variables d'écrouissage}
\txb{125}{2}{11}{
	\begin{itemize}
		\item Variables d'écrouissages $\equiv$ variables internes
		\item Utilisées pour décrire l'évolution de $f\pscp$
		\item $\chih$: variables d'écrouissage statiques\\ $\to$ collection des variables scalaires, vectorielles, tensorielles \beq{\chih=\pbp{R,\theta,...,\backstress,\tens{Y}{}{},...}}{chi-static}
		\item $\etah$: variables d'écrouissage cinématiques \\$\to$ collection des variables scalaires, vectorielles, tensorielles
		\item Modèle de Armstrong-Frederick $ \etah=\pbp{\kbackstress,r}$:
			\bsys[\chih\equiv][
		& \backstress\pdp{\kbackstress}=\frac{2}{3}C\kbackstress&\\
		&R\pdp{r} = R_{\infty}\pdp{1- \exp\pdp{-br}} & ][]
	\end{itemize}
}
\txb{38}{88}{62}{
\igwf{hardening-scheme}{40mm}
}
\end{frame}


\begin{frame}{Résolution par étapes}
\txb{123}{2}{11}{
	Évolution continue de $f$ en temps : $f\pdp{t+\delta t}\approx f\pdp{t}+\delta f$
	\bsys[\text{si  } f\pdp{t} \equiv 0:][
	& \delta f < 0 \to f\pdp{t+\delta t} < 0& \text{déchargement élastique}\\
	& \delta f = 0 \to f\pdp{t+\delta t} = 0& \text{chargement plastique}
	][cond_consistency_ep]
	\vspace{-3mm}
	\begin{enumerate}
		\item Hypothèse 0: $\dstrainpl=\tens{0}{}{} \to \dstress[][trial]=\Del : \dstrain$\\
		\item Vérifier la valeur de $f\pdp{\stress+\dstress[][trial];\chih}\leq 0$ 
		\item si $f\pdp{\stress+\dstress[][trial];\chih}> 0\to$ correction: $f\pdp{\stress+\dstress[][];\chih+\dchih}=0$
	\end{enumerate}
	\centering
	\ighf{nl_int_1step}{25mm}
	\ighf{nl_int_1step_unl}{25mm}
}
\end{frame}

\begin{frame}{Écrouissage}
\txb{123}{2}{11}{
	\bsys[][
	& f\pdp{t}=0 \text{ and } \gradfstress : \dstress[][trial] > 0 & \text{écrouissage positif}\\
	& f\pdp{t}=0 \text{ and } \gradfstress : \dstress[][trial] = 0 & \text{plasticité parfaite ($f=f_{ult}$)}\\
	& f\pdp{t+\delta t}=0 \text{ and } \gradfstress : \dstress[][trial] < 0 & \text{écrouissage négatif}\\
	& f\pdp{t+\delta t}<0 \text{ and } \gradfstress : \dstress[][trial] < 0 & \text{écrouissage négatif}
	][hardening]
	\centering
	\ighf{hardening}{32mm}
	\ighf{hardening-vs-perfect}{32mm}
}
\end{frame}

\begin{frame}{Loi d'écoulement}

\txb{95}{28}{11}{
\bsys[][
& \dstrainpl=\dplm.\gradgstress, \quad\detah =- \dplm.\gradghidd & \text{non-associée}\\
& \dstrainpl=\dplm.\gradfstress,\quad  \detah =- \dplm.\gradfhidd & \text{associée} (f=g) 
][flow-rule]
}
\txb{123}{2}{40}{
	\beq{\plm=\int_{0}^{t}\sqrt{\frac{2}{3}\dco{\dstrainpl}}dt=}{plastic_multiplier}
\igwf{flow-rule}{80mm}\igwf{non-associated-fr}{40mm}
}
\end{frame}

\begin{frame}{Loi d'écoulement}
\txb{123}{2}{11}{
	Modèle de Armstrong-Frederick : $\lambda,\mu,C,\kappa,b, R_\infty,\sigma_{yld}$ : paramètres du modèles\\
	Potentiel plastique: \beq{g\pscp=f\pscp+\underbrace{\frac{3}{4}\frac{\kappa}{C}\backstress:\backstress}_{\text{\textbf{fading memory}}} }{AF-pp}
	\beq{\dstrainpl  = \dplm\gradgstress = \dplm\gradfstress=\dplm\frac{3}{2}\frac{\deviator-\backstress}{\fvm}}{AF-depl}
	\bsys[\detah\equiv][
	& \dkbackstress = \dplm\gradgX =\dplm\frac{3}{2}\frac{\deviator-\backstress}{\fvm}-\dplm\frac{3\kappa}{2C}\backstress\\
	& \delta r = -\dplm \frac{\partial g}{\partial R}=-\dplm \frac{\partial f}{\partial R}=\dplm \to r\equiv  \lambda
][AF-flow-rule]
}
\end{frame}

\begin{frame}{Évolution du seuil de plasticité}
\txb{125}{2}{11}{
\beq{\dchih=\gradchieta:\detah}{grad_hidd_var}
\centering
\ighf{linear-kinematic-prager}{32mm}\ighf{nlinear-kinematic-prager}{32mm}\ighf{isotropic-hardening}{32mm}
}
\end{frame}

\begin{frame}{Évolution du seuil de plasticité}
\txb{125}{2}{11}{
Modèle de  Armstrong-Frederick:
\bsys[\dchih\equiv][
& \dbackstress\pdp{\kbackstress}=\frac{2}{3}C\dkbackstress = \pdp{ C\frac{\deviator-\backstress}{\fvm}-\kappa\backstress}\dplm\\
&\delta R\pdp{r} = b\pdp{R_{\infty}-R\pdp{r}} \dplm
][AF-flow-rule]
}
\end{frame}

\begin{frame}{Problème d'optimisation}
\txb{125}{2}{11}{
	Conditions de Karush–Kuhn–Tucker conditions:
\bsys[\mathcal{K}:][
&f\pscp\leq 0, \\
&\plm \geq 0, & \forall f\pscp \in  \mathbb{S} \times \mathbb{K}  \\
&\dplm f\pscp = 0
][KKT_consistency]
$\dplm$ est calculé à partir des conditions de consistance élastoplastique:
$$\text{if } f= 0, \to \delta f = 0 \to f\pdp{t+\delta t} = 0,\quad \text{chargement plastique}$$\vspace{-5mm}
%\beq{\delta f = \gradfstress:\underbrace{\Del:\pdp{\dstrain-\dplm\gradgstress}}_{\dstress}+\gradfhidd:\underbrace{\gradchieta}_{\dchih}:\etah}{cond_consistency_1}
\beqh[\delta f = \gradfstress:\underbrace{\Del:\pdp{\dstrain-\dplm\gradgstress}}_{\dstress}+\gradfhidd:\underbrace{\gradchieta:\gradghidd\dplm}_{\dchih=\gradchieta:\detah}=0][cc][][ABC]
}
\end{frame}

\begin{frame}{Module d'écrouissage}
\txb{125}{2}{11}{
	En résolvant l'\hyperlink{ABC}{\Cref{eq:cc}}, on trouve l'incrément du multiplicateur plastique:
\beq{\delta f = 0\to \dplm=\frac{\gradfstress:\Del:\dstrain}{\underbrace{\gradfstress:\Del:\gradgstress}_{\mathcal{H}_{c}\text{=module d'ecrouissage critique}} - \underbrace{\gradfhidd:\gradchieta:\gradghidd}_{\mathcal{H}\text{=module d'ecrouissage}}}}{plm_sol}
Modèle de Armstrong-Frederick:
}

\txb{60}{2}{50}{
{\scriptsize
\bsys[][
& \gradfX=-\frac{3}{2}\frac{\deviator-\backstress}{\fvm};\frac{\partial f}{\partial R}=-1\\
&  \tens{\delta X}{}{}=\pdp{ C\frac{\deviator-\backstress}{\fvm}-\kappa\backstress}\dplm\\
&\delta R = b\pdp{R_{\infty}-R\pdp{r}} \dplm
][AF-Hcomp]}
}

\txb{60}{62}{55}{
	{\scriptsize
	\bsys[][
	& \Hent = C-\frac{3\kappa}{2}\frac{\pdp{\deviator-\backstress}:\backstress}{\fvm}+b\pdp{R_{\infty}-R}\\
	& \Hent[c] = 3\mu
	][AF-moduleH]
}
}
\end{frame}


\begin{frame}{Tenseur de rigidité élastoplastique}
\txb{125}{2}{11}{
À l'aide de l'\Cref{eq:plm_sol}, l'incrément élastoplastique de l'état de contrainte peut s'écrire sous forme:
\beq{\dstress = \Del:\pdp{\dstrain+\underbrace{\frac{\gradfstress:\Del:\dstrain}{\Hent-\Hent[c]}}_{-\dplm}\gradgstress}}{dstress-ep}
Le tenseur de rigidité élastoplastique est identifiée :
\beq{\Dep=\Del:\pdp{\IDD+\frac{\gradgstress\otimes\Del:\gradfstress}{\Hent-\Hent[c]}}}{Dep}

\beq{\dstress=\Dep:\dstrain}{dstressDepdstrain}

}
\end{frame}
