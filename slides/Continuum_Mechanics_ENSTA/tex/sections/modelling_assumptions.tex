\subsection{Hypothèses fondamentales}
\label{subsec:modelling_hypothesis}
\framecard{Hypothèse fondamentales}
\begin{frame}{Généralités}
\begin{enumerate}
\item Relation entre contraintes et déformations (et/ou leur taux) \\
$ \stress \longleftrightarrow \strain$ ??
\item Nécessité d'un état de contraintes (déformations) homogènes, sans
localisation des déformations
\item Nécessité de définir les domaines de comportement\\
Élasticité vs. Plasticité vs. Viscosité ??
\item Nécessité des expériences au laboratoire
\item Représentatif des conditions in-situ
\item Chemin de chargement
\item Rôle de l'existence de l'eau
\end{enumerate}
\end{frame}

\begin{frame}{Hypothèses fondamentales}
\begin{itemize}
	\item Cinématique linéarisée : formulation en petites déformations \\ $$\strain=\frac{1}{2}\pdp{\gradt{u}{x}{}+\gradt{u}{x}{T}}$$
	$\vct{u}{}{}$ continu en espace/temps
	\item Milieux saturé d'eau (pression d'eau $u_w$) $\to$ contraintes effectives:
	\beq{\stress = u_w\ID+\stress[][,]}{effective-stress}
	$\stress[][,]$ et $u_w$ continus en espace/temps
\item Pas d'influence de la vitesse de sollicitation:
\beq{\stress[][,] = \mathcal{F}\pdp{\strain,\chih},\quad \chih \text{ variables statiques d’écrouissage}}{stress-strain-generic}
\item Conditions isothermes $$T=const.$$
\end{itemize}
\end{frame}

\begin{frame}{Domaines de comportement}
\bfc{\subfh{strain-stress-traction-ex}{30mm}{}\subfh{elastic-domain}{30mm}{}}{\beq{\strain=\strainel+\strainpl}{strainelpl}}{assumptions}
\only<1>{
\begin{itemize}
	\item $\strainel$ sont réversibles (récupérées après décharge)
	\item $\strainpl$ sont irréversibles (non récupérées après décharge)
\end{itemize}
}
\only<2>{
\begin{itemize}
	\item Comportement Élastique (réversible) : $f<0$
	\item Écoulement Plastique (irréversible) : $f=0$
	\item $f$ définie dans l'espace de contrainte mais elle varie avec écoulement plastique (écrouissage)
\end{itemize}
}
\end{frame}

\begin{frame}{Questions ouvertes}
\begin{itemize}
	\item Comment définir le limite élastique en fonctions de contraintes $f\pdp{\stress}$?
	\item Quel repère utiliser?
\item Comment définir différent mécanisme (volumétrique, cisaillement etc)?
\item Comment définir l'écrouissage (évolution non-linéaire de $f$) $$\delta f = df\pdp{\stress,\chih}=\gradfstress:\dstress+\gradfhidd:\dchih$$
\end{itemize}
\end{frame}

\begin{frame}{Exemple d'écrouissage cyclique}
\begin{textblock}{125}(2,10)
	\begin{figure}
		\begin{center}
			\animategraphics[height=65mm,keepaspectratio,autoplay,loop,poster,loop]{1}{deviator_plane_}{0}{3}
		\end{center}
	\end{figure}
	\textbf{Limit strength :} \fbox{$\boldsymbol{\tau_{max} = \frac{C}{\kappa}+\sigma_{yld}}$} 
\end{textblock}
\end{frame}