\subsection{Modèles EP pour les argiles}
\label{subsec:ep-models-clays}
\framecard{Modèles EP pour les argiles}
\begin{frame}{Comportement expérimental}
\txb{125}{0}{11}{
	\centering
\igwf{comportement-typique}{125mm}
}
\end{frame}

\begin{frame}{Ingrédients de modélisation}
\txb{125}{2}{11}{
\begin{itemize}
	\item<+-> comportement volumétrique/déviatorique couplés
	\begin{enumerate}
		\item<+->  volumétrique
		\begin{itemize}
			\item compression isotrope
			\item compression oedométrique (conditions à repos, in-situ$\to$ consolidation)
			\item variables intéressées: $\stressm'(=p')$ et $\evol$ (ou index des vides $e$)
		\end{itemize} 
		\item<+->  déviatorique
		\begin{itemize}
			\item comportement sous charge triaxial/multiaxial
			\item variables intéressées: $q-p'$ et $\edev,\evol$ ($\edev=\sqrt{\frac{4}{3}J_2\pdp{\strain}}$)
		\end{itemize}
	\end{enumerate}
	\item<+-> rôle de la pression d'eau interstitielle $u_w$:
	\begin{enumerate}
		\item<+->  conditions drainée: $\stress=\stress[][']+u_w\ID$ (Terzaghi)
		\item<+->  conditions non-drainée: $\devol=0$ (sollicitations dynamiques rapides)
	\end{enumerate}
\item<+-> relation élastoplastique (conditions triaxiales)
\beq{
\triaxdstress=\Deptriax\triaxdstrain \quad  \triaxdstrain=\Ceptriax\triaxdstress 
}{}
\end{itemize}
}
%\bfc{\igwf{example-oedo}{60mm}}{\cite{Book_Bardet_1997_Experimental_soil_mechanics}}{comport-repere}
\end{frame}


\begin{frame}{\hypertarget{frame:comp-iso}{Compression isotrope}}
\bfc{\ighf{compression-iso}{35mm}\ighf{example-iso}{40mm}{}}{\cite{Book_Bardet_1997_Experimental_soil_mechanics}}{example-oedo}
$\evol=3\depsil_1=Tr\pdp{\strain}$ déformation volumique
\end{frame}

\begin{frame}{Compression odeométrique}
\ighf{chemin-oedometrique}{35mm}\\
En élasticité: $\devol=\delta\depsil_1=\delta\depsil_a=\frac{\delta H}{H}=3K\dsigma_1$
\bfc{\ighf{example-oedo}{30mm}\ighf{compression-e-logp-davis}{30mm}}{\cite{Book_Bardet_1997_Experimental_soil_mechanics}}{example-oedo}
\end{frame}

\begin{frame}{Comportement volumétrique des argiles}
\txb{60}{0}{11}{
	\centering
	\ighf{p-e-plane}{28mm}
}
\txb{60}{60}{11}{
	\begin{itemize}
		\item Comportement élastique non linéaire (linéaire en $\log p'$)
\item $\lambda$ et $\kappa$ dans le plan $\ln p'-e $
\item  $C_c$ et $C_s$ dans le plan $\log p'-e $ et $\log \dsigma_v'-e$
\item $ OCR=\frac{\dsigma'_{vc}}{\dsigma'_v}$ ou $OCR=\frac{p'_c}{p'}$ 
	\end{itemize}
}
\txb{123}{2}{40}{
\begin{itemize}
\item<+-> $\devol=-\frac{\delta e}{1+e_0}=-\frac{\delta e}{\Gamma}$ {\scriptsize (compression positive)}
\item<+-> $e_0$ index de vide à repos {\scriptsize(ou conventionnellement, correspondant à $p' $=1 kPa)}
\item<+-> $\devol = \devol[el]+\devol[pl]$, $\delta e = \delta e^{el}+\delta e^{pl}$
\item<+-> Consolidation normale [virgin consolidation]: \\ 
$\to \delta e = -\lambda \delta \pdp{\ln p'}=  -\lambda \frac{\delta p'}{ p'},\quad \nu = \Gamma-\ln\pdp{\frac{p'}{p'_0}} $ {\scriptsize $\pdp{\Gamma=1+e_0}$}
\item<+-> Décharge élastique [swelling line]: $\delta e ^{el} = -\kappa \frac{\delta p'}{p'},\delta e ^{pl}=-\pdp{\lambda-\kappa}\frac{\delta p'}{p'}$\\
 \vspace{-3mm}
$\to\nu=\nu_c-\kappa\ln\pdp{\frac{p'}{p'_c}}$ \\
$e_c=e_{c0}-\lambda\ln\pdp{\frac{p'_c}{p'_{c0}}}$ ou $\nu_c=\nu_{c0}-\lambda\ln\pdp{\frac{p'_c}{p'_{c0}}}$
\end{itemize}
}
\end{frame}

\begin{frame}{\hypertarget{frame:triax-revo}{Chargement Triaxial}}
\only<1>{\ighf{example-triax-rev}{65mm}
	$$\rho=\sqrt{2J_2\pdp{\deviator}}=\sqrt{\frac{2}{3}}q$$}
\only<2>{
	\txb{65}{0}{35}{
	\igwf{options-triax-rev}{65mm}
}
\txb{60}{65}{35}{
	\begin{itemize}
		\item compression triaxiale: $q>0, \dsigma_a>\dsigma_r$
		\item extension triaxiale: $q<0, \dsigma_a<\dsigma_r$
	\end{itemize}
}
\txb{125}{2}{11}{
	Comportement déviatorique dépende de:
	\begin{itemize}
		\item la consolidation (isotrope ou oedometrique) $\to$ argiles
		\item densité $\to$ sables
	\end{itemize}
}
}
\end{frame}

\begin{frame}{Chargement triaxial}
\txb{125}{0}{5}{
\bfc{\subfw{clays-NC}{60mm}{}
\subfw{loose-sands-triax}{60mm}{}
}{ Compression triaxiale monotone : (a) argiles normalement consolidées; (b) sables lâches (\cite{Ziani_Biarez_1990} d'après Lopez-Caballero)}{triax-loose}
}
\end{frame}

\begin{frame}{Ligne d'état critique}
\bfc{\ighf{normal-consolidation-line-3d}{65mm}}{Ligne d'état critique dans l'espace $q-p'-e$}{normal-consolidation-line-3d}
\end{frame}

\begin{frame}{Modèle de Mohr Coulomb}

	\txb{58}{2}{33}{
		\bfc{\igwf{coulomb}{58mm}}{\cite{Han_et_al_2006}}{coulomb}
	}
	\txb{100}{0}{10}{
		\beq{f\pdp{\stress[][']}=\frac{\vert\dsigma_I-\dsigma_{III}\vert}{2}-\frac{\dsigma_I+\dsigma_{III}}{2}\sin\phi'-c'\cos\phi<0}{coulomb}
\beq{g\pdp{\stress[][']}=\vert\dsigma_I-\dsigma_{III}\vert-\pdp{\dsigma_I+\dsigma_{III}}\sin\psi<0}{coulomb}
	\txb{65}{60}{35}{
		\begin{itemize}

			\item Critère de Coulomb:
			\begin{itemize}
				\item $\phi'$ angle de frottement
				\item $c'$ cohésion drainée  apparente 
			\end{itemize}
		\item angle de dilatance $\psi$ ($f\neq g$)
			\item élastoplasticité parfaite (pas d'écrouissage)
		\end{itemize}
	\centering
	\igwf{perfect-ep}{27mm}
	}
}
\end{frame}

\begin{frame}{Modèle de Mohr-Coulomb}
	\txb{60}{2}{11}{
		\igwf{MC-model}{60mm}
	}
\txb{65}{60}{11}{
\beq{f\pdp{\stress[][']}=\vert q\vert -Mp'-N\leq 0}{MC-failure}
\bsys[][
& \dsigma_{I} = \frac{3p'+2q}{3},\dsigma_{III} = \frac{3p'-q}{3}\\
&M = \frac{6\sin\phi'}{3sgn\pdp{q}-\sin\phi'}\\
& N = \frac{6c'\cos\phi'}{3sgn\pdp{q}-\sin\phi'}][MC-pq-properties]
}
\txb{125}{2}{70}{
\beq{g\pdp{\stress[][']}=\vert q \vert- M_gp'}{g-MC}
\beq{M_g=\frac{6\sin\psi}{3sgn\pdp{q}-\sin\psi}}{Mg-MC}
}
\end{frame}


\begin{frame}{Modèle de Mohr-Coulomb en 3D}
\txb{123}{2}{22}{
	
\bfc{\centering\igwf{mohr-coulomb-3d}{40mm}}{\cite{Panteghini_Lagioia_2014}}{mohr-coulomb-3d}
}
\txb{123}{2}{11}{
	\beq{f\pdp{\stress[][']}\frac{\vert q\vert}{3} \pdp{\sqrt{3}\cos\theta-\sin\theta\sin\phi'}-p'\sin\phi'-2c'\cos\phi'\leq 0}{MC-3D}
}
\end{frame}

\begin{frame}{Rôle de la dilatance}
\txb{123}{2}{7}{
	\only<1>{
	\bfc{\igwf{dilatancy-concept}{90mm}}{Essaie de cisaillement direct. (a) Courbes $\tau-u$; (b) Courbes $v-u$; (c) mécanisme d'inter-locking}{dilatancy-concept}
}
	\only<2>{
	\bfc{\subfw{dilatancy-curves-type}{65mm}{}	\\ \subfw{dilatancy-angles-mohr}{65mm}{}
}{Essaie de cisaillement direct: (a) Courbes typiques; (b) loi d'écoulement associée ($\psi=\phi'$) et non-associé ($\psi=\neq\phi'$)}{dilatancy}
}
}
\end{frame}

\begin{frame}{Comportement volumique}
\txb{83}{2}{11}{
Le modèle de Mohr-Coulomb est trop simple:\\
	\begin{itemize}
		\item<+-> Conditions drainées $\to$ comportement volumique:
		\begin{enumerate}
			\item phase élastique : $\dpeff\uparrow \devolel\uparrow$
			\item phase élastoplastique dépende du $OCR$
			\begin{itemize}
				\item contraction (NC ou  $OCR<4$)
				\item dilatance ( $OCR>4$): mesurée comme $\frac{\devolpl}{\dedevpl}$
			\end{itemize}
		\end{enumerate}
	
		\item<+-> Conditions non-drainées $\devol=0$:
		\begin{enumerate}
			\item phase élastique : 
			\begin{itemize}
				\item 	$\devol=\devolel=0=\Celpp\dpeff \pdp{\Celpq=0} \to \delta p=\duw\uparrow $
			\end{itemize}
			\item phase élastoplastique : 
			\begin{itemize}
				\item $OCR<4\to \duw>0$
				\item $OCR>4\to \duw >0 $ et $\duw<0$
				\item $\dpeff = \Ceppp \cancelto{0}{\devol}+\Ceppq\dedev=\delta p+\duw$
				\item $\dq = \Cepqp \cancelto{0}{\devol}+\Cepqq\dedev$
			\end{itemize}
			\end{enumerate}
	\end{itemize}
}
\txb{40}{85}{20}{
	\igwf{normal-consolidation-line-3d}{40mm}
	%\igwf{cap-model}{40mm}
}
\end{frame}

\begin{frame}{Compression isotrope}
	\txb{125}{2}{11}{
	\begin{itemize}
		\item Comportement élastique : \beq{\frac{\dpeff}{p '}=\frac{1+e}{\kappa}\devolel=\frac{\nu}{\kappa}\devolel}{dp-devolel-iso}
		\item Comportement élastoplastique : \beq{\frac{\dpeff}{p '}=\frac{1+e}{\lambda-\kappa}\devolpl=\frac{\nu}{\lambda-\kappa}\devolpl}{dp-devolpl-iso}
	\end{itemize}
}

\txb{40}{85}{50}{
	\igwf{normal-consolidation-line-3d}{40mm}
	%\igwf{cap-model}{40mm}
}
\end{frame}

\begin{frame}{Rôle de la dilatance}
\txb{125}{2}{11}{
	Incrément de travail élastoplastique (constante le long l'essai): \beq{\delta W^{ep}=\stress[][']:\dstrainpl=p'\devolpl+q\dedevpl}{dWep}
	À l'état critique en compression (argile normalement consolidée):
	\bsys[][
	& \devol\approx\devolpl=0, \quad \dq=M\dpeff\\
	& \delta W^{ep} = M p' \dedevpl=p'\devolpl+q\dedevpl
	][CS-conditions]
}
\txb{75}{0}{55}{
	\bsys[\frac{\devolpl}{\dedevpl}\equiv][
	& M-\frac{q}{p'} > 0 & \text{compression}\\
	& M-\frac{q}{p'} = 0 & \text{CS}\\
	& M-\frac{q}{p'} < 0 & \text{dilatance}
	][conditions-dilatancy-CS]
}
\txb{40}{85}{50}{
	\igwf{normal-consolidation-line-3d}{40mm}
	%\igwf{cap-model}{40mm}
}
\end{frame}

\begin{frame}{Cam-Clay (CCM) et version modifiée(MCCM)}
\txb{125}{2}{11}{
\begin{itemize}
	\item Fonction de charge: 
	\bsys[f\pdp{\stress}\equiv][
	& q+Mp'\ln \pdp{\frac{p'}{p'_c}}\leq 0 & \text{CCM}\\
	& \frac{q^2}{p'^2}+M ^2\pdp{1-\frac{p'_c}{p'}}\leq 0 & \text{MCCM}
	][CCM-failure]
\end{itemize}
}
\txb{50}{0}{50}{
\igwf{CCM-failure}{50mm}
}
\txb{75}{50}{55}{
$p'_c$ : pression de pré-consolidation 
\begin{itemize}
	\item[$\to$] contrôle la taille de la surface 
	\item[$\to$] paramètre d'écrouissage (voir \Cref{eq:dp-devolpl-iso}) \beq{p'_c = p'_{c0}\exp\pdp{\frac{\nu}{\lambda-\kappa}\evolpl}}{pc-evol}
\end{itemize}
}
\end{frame}
