\subsection{Déformations}
\label{sec:deformations}
\framecard{Déformations}
\begin{frame}{Description du mouvement}
\only<1>{
\bfc{\igwf{configuration_mapping}{110mm}}{Configurations}{}
}
\only<2-3>{
	Vecteur position: cartographie des points matériels $M$ dans repère Euclidien $\mathcal{E}$ muni d'un repère (observateur) $\mathcal{R}=\pdp{\ibv[1],\ibv[2],\ibv[3]}$
	
	\beq{\pv=\vct{p}{}{}\pdp{M} \in \Omega_0,\quad \pv=\sum_{n=1}^{3}p_k\ibv[k]}{position} 
	
	\beq{\xv=\vct{f}{}{}\pptp \in \Omega_t,\quad\forall \pptp \in \pdp{\Omega_0,\mathbb{I}_t} }{xpu}
}
\only<2>{
	\begin{enumerate}
		\item Trajectoires: $\mathcal{T}\pdp{M}=\pbp{\xv, \exists t'\Big\vert\quad \xv=\vct{f}{}{}\pdp{\pv;t'}}$
		\item Caractériser petites changement de trajectoires en temps
		\item Caractériser petites changement de position en espace
		\item La matière ne peut pas disparaître
	\end{enumerate}
}
\only<3>{
	\vspace{-2mm}
	\begin{enumerate}
		\item \vct{f}{}{}: transformation injective (deux points matériels peuvent pas occuper le même volume) $ \vct{p}{1}{}\neq\vct{p}{2}{} \Rightarrow \vct{x}{1}{}\neq\vct{x}{2}{}$
		\item Il existe sa transformation inverse $\vct{g}{}{}:\Omega_t\to\Omega_0$
		\item $\vct{f}{}{}\in \mathcal{C}^1\pdp{\Omega_k,\Delta_T}$ (par morceaux en espace/temps)
		\item Le trièdre reste droit (matière ne peut pas disparaître): $ \det\pdp{\vct{e}{1}{},\vct{e}{2}{},\vct{e}{3}{}}>0 \Leftrightarrow\det\pdp{\ibv[1],\ibv[2],\ibv[3]}>0,\quad \vct{e}{k}{}=\vct{f}{}{}\pdp{\vct{i}{k}{},t}$
	\end{enumerate}
}
\end{frame}

\begin{frame}{Déplacement et Gradient}
\only<1>{
\bfc{\igwf{configuration_gradient}{110mm}}{Tenseur gradient [Credits: G. Puel]}{}
}
\only<2>{
\begin{itemize}
	\item Vecteur déplacement:
	\beq{\uv\pptp=\xv\pptp-\pv=\vct{f}{}{}\pptp-\pv}{}
	\item Tenseur gradient de la transformation: \beq{\vct{dx}{}{}=\Fgrad.\vct{dp}{}{},\quad \vct{du}{}{}=\pdp{\Fgrad-\ID}.\vct{dp}{}{}}{dxFdp} 
\end{itemize}
	\textit{Les vecteurs tangents des matériaux cartographient les vecteurs tangents spatiaux via le gradient de déformation.}	
\beq{\Fgrad=\cartFgrad}{Fexp}
\beq{\Fgrad=\curvFgrad}{Fcurv}\\ \hspace{-2mm}\vspace{5mm} coordonnées curvilignes $\xi_i$
}
\end{frame}

\begin{frame}{Larges Déformations [Large Strains]}
\only<1>{
	\txb{122}{2}{12}{
	\bfc{
		\begin{minipage}{55mm}
			\igwf{configuration_green_lagrange}{55mm}
		\end{minipage}
		\begin{minipage}{60mm}
			\igwf{dx1dx2_dp1dp2}{60mm}\\ \igwf{angular-distorsion}{25mm}
		\end{minipage}
}{Mesure de déformation [Credits: G. Puel]: Allongement [top] - Distorsion angulaire [down]}{}
	}
}
\only<2>{
	\txb{125}{0}{9}{
		\hspace{-2mm}
\begin{itemize}
	  \itemsep0em 
	\item Tenseur des déformations de Green-Lagrange:\vspace{-3mm} \beq{\Estrain=\FEstrain=\UEstrain}{Estrain}\vspace{-3mm}
		\item  Allongement de la fibre $\ell_p\vct{i}{p}{}$:
		\beq{\frac{\norm{\vct{dx}{}{}}^2-\norm{\vct{dp}{}{}}^2}{\norm{\vct{dp}{}{}}^2}=\frac{\ell_x^2-\ell_p^2}{\ell_p^2}=2\pscl{\vct{i}{p}{}}{\Estrain. \vct{i}{p}{}} = 2 \mathbb{E}_{pp} }{dx2dp2} \vspace{-3mm}
		\item Distorsion angulaire des fibres $\ell_{p1}\vct{i}{p1}{}$ $\ell_{p2}\vct{i}{p2}{}$:
	\vspace{-5mm}
		\beq{
			\begin{split}
			& \sin\pdp{\gamma_{x12}} =\sin\pdp{\frac{\pi}{2}-\alpha_{x12}}= \cos\pdp{\alpha_{x12}} =  \\
			& = \frac{\pscl{\pdp{2\Estrain+\ID}.\vct{i}{p1}{}}{\vct{i}{p2}{}}}{\sqrt{
					\pscl{\pdp{2\Estrain+\ID}.\vct{i}{p1}{}}{\vct{i}{p1}{}}
				}\sqrt{
					\pscl{\pdp{2\Estrain+\ID}.\vct{i}{p2}{}}{\vct{i}{p2}{}}	
		}}   = \frac{2\mathbb{E}_{12}+1}{\sqrt{\pdp{2\mathbb{E}_{11}+1}\pdp{2\mathbb{E}_{22}+1}}}
	\end{split}
}{sinEstrain}
	\end{itemize}
}
}
\only<3>{
\bfc{\ighf{example-green-lagrange}{65mm}}{Credits: G. Puel.}{example-green-lagrange}
}
\end{frame}

\begin{frame}{\hypertarget{frame:NAN}{Transformation de surface/volume}}
Formule de Nanson \textcolor{blue}{[NAN]}:
\setcounter{equation}{9}

\begin{itemize}
	\item Variation locale de surface $dS_0\Nv$:
	\beq{ds_t\nv=dS_0 det\pdp{\Fgrad}\FgradTinv\Nv}{dS}
	\item Variation locale de volume $dV_0$:
	\beq{dv_t = det\pdp{\Fgrad} dV_0}{dVolJ} (propriété 4 de \vct{f}{}{}: le trièdre doit rester droit! $J=det\pdp{\Fgrad}$)
\end{itemize}
\end{frame}


\begin{frame}{Petites Déformations [Small Strains]}
\txb{110}{0}{11}{
\begin{itemize}
	\item Petits déplacements/Petites déformations:	\beq{\max_{\pxtp}\norm{\uv} \ll \mathcal{L} \pdp{\xv\approx\pv},  \max_{\pxtp}\norm{\gradt{u}{p}}^2=\max_{\pxtp}\gradpUT:\gradpU \ll 1}{petitDepDef} 
	\item Tenseur des petites déformations:
	\beq{\Estrain\approx\strain=\frac{1}{2}\pdp{\gradt{u}{x}+\gradtT{u}{x}} =\cartEspilon}{smallStrain}
	 $$\frac{\Delta \ell}{\ell_p} = \pscl{\vct{i}{p}{}}{\depsil \vct{i}{p}{}}=\depsil_{pp},\quad \frac{\Delta\gamma_{12}}{2}=\pscl{\vct{i}{1}{}}{\depsil \vct{i}{2}{}}=\depsil_{12}$$
	 \item Variation locale de volume:
	 {\scriptsize
	 \beq{\evol=\frac{\Delta v_t}{V_0}=\frac{\ell_{p1}+\Delta \ell_1}{\ell_{p1}}\frac{\ell_{p2}+\Delta \ell_2}{\ell_{p2}}\frac{\ell_{x3}+\Delta \ell_{p3}}{\ell_{p3}}\approx \frac{\Delta \ell_1}{\ell_{p1}}+ \frac{\Delta \ell_2}{\ell_{p2}}+\frac{\Delta \ell_{p3}}{\ell_{p3}}\approx Tr\pdp{\depsil}}{epsvol}\\ \vspace{-2mm}
 $\Delta \ell_{pi}= o\pdp{\ell_{pi}}$}
\end{itemize}
}
\txb{15}{110}{11}{
\igwf{small-displ}{10mm}
}
\end{frame}